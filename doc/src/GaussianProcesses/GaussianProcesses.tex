%%
%% Automatically generated file from DocOnce source
%% (https://github.com/hplgit/doconce/)
%%

% #define PREAMBLE

% #ifdef PREAMBLE
%-------------------- begin preamble ----------------------

\documentclass[%
oneside,                 % oneside: electronic viewing, twoside: printing
final,                   % draft: marks overfull hboxes, figures with paths
10pt]{article}

\listfiles               %  print all files needed to compile this document

\usepackage{relsize,makeidx,color,setspace,amsmath,amsfonts,amssymb}
\usepackage[table]{xcolor}
\usepackage{bm,ltablex,microtype}

\usepackage[pdftex]{graphicx}

\usepackage[T1]{fontenc}
%\usepackage[latin1]{inputenc}
\usepackage{ucs}
\usepackage[utf8x]{inputenc}

\usepackage{lmodern}         % Latin Modern fonts derived from Computer Modern

% Hyperlinks in PDF:
\definecolor{linkcolor}{rgb}{0,0,0.4}
\usepackage{hyperref}
\hypersetup{
    breaklinks=true,
    colorlinks=true,
    linkcolor=linkcolor,
    urlcolor=linkcolor,
    citecolor=black,
    filecolor=black,
    %filecolor=blue,
    pdfmenubar=true,
    pdftoolbar=true,
    bookmarksdepth=3   % Uncomment (and tweak) for PDF bookmarks with more levels than the TOC
    }
%\hyperbaseurl{}   % hyperlinks are relative to this root

\setcounter{tocdepth}{2}  % levels in table of contents

% --- fancyhdr package for fancy headers ---
\usepackage{fancyhdr}
\fancyhf{} % sets both header and footer to nothing
\renewcommand{\headrulewidth}{0pt}
\fancyfoot[LE,RO]{\thepage}
% Ensure copyright on titlepage (article style) and chapter pages (book style)
\fancypagestyle{plain}{
  \fancyhf{}
  \fancyfoot[C]{{\footnotesize \copyright\ 2018-2019, Christian Forssén. Released under CC Attribution-NonCommercial 4.0 license}}
%  \renewcommand{\footrulewidth}{0mm}
  \renewcommand{\headrulewidth}{0mm}
}
% Ensure copyright on titlepages with \thispagestyle{empty}
\fancypagestyle{empty}{
  \fancyhf{}
  \fancyfoot[C]{{\footnotesize \copyright\ 2018-2019, Christian Forssén. Released under CC Attribution-NonCommercial 4.0 license}}
  \renewcommand{\footrulewidth}{0mm}
  \renewcommand{\headrulewidth}{0mm}
}

\pagestyle{fancy}


\usepackage[framemethod=TikZ]{mdframed}

% --- begin definitions of admonition environments ---

% Admonition style "mdfbox" is an oval colored box based on mdframed
% "notice" admon
\definecolor{mdfbox_notice_background}{rgb}{1,1,1}
\newmdenv[
  skipabove=15pt,
  skipbelow=15pt,
  outerlinewidth=0,
  backgroundcolor=mdfbox_notice_background,
  linecolor=black,
  linewidth=2pt,       % frame thickness
  frametitlebackgroundcolor=mdfbox_notice_background,
  frametitlerule=true,
  frametitlefont=\normalfont\bfseries,
  shadow=false,        % frame shadow?
  shadowsize=11pt,
  leftmargin=0,
  rightmargin=0,
  roundcorner=5,
  needspace=0pt,
]{notice_mdfboxmdframed}

\newenvironment{notice_mdfboxadmon}[1][]{
\begin{notice_mdfboxmdframed}[frametitle=#1]
}
{
\end{notice_mdfboxmdframed}
}

% Admonition style "mdfbox" is an oval colored box based on mdframed
% "summary" admon
\definecolor{mdfbox_summary_background}{rgb}{1,1,1}
\newmdenv[
  skipabove=15pt,
  skipbelow=15pt,
  outerlinewidth=0,
  backgroundcolor=mdfbox_summary_background,
  linecolor=black,
  linewidth=2pt,       % frame thickness
  frametitlebackgroundcolor=mdfbox_summary_background,
  frametitlerule=true,
  frametitlefont=\normalfont\bfseries,
  shadow=false,        % frame shadow?
  shadowsize=11pt,
  leftmargin=0,
  rightmargin=0,
  roundcorner=5,
  needspace=0pt,
]{summary_mdfboxmdframed}

\newenvironment{summary_mdfboxadmon}[1][]{
\begin{summary_mdfboxmdframed}[frametitle=#1]
}
{
\end{summary_mdfboxmdframed}
}

% Admonition style "mdfbox" is an oval colored box based on mdframed
% "warning" admon
\definecolor{mdfbox_warning_background}{rgb}{1,1,1}
\newmdenv[
  skipabove=15pt,
  skipbelow=15pt,
  outerlinewidth=0,
  backgroundcolor=mdfbox_warning_background,
  linecolor=black,
  linewidth=2pt,       % frame thickness
  frametitlebackgroundcolor=mdfbox_warning_background,
  frametitlerule=true,
  frametitlefont=\normalfont\bfseries,
  shadow=false,        % frame shadow?
  shadowsize=11pt,
  leftmargin=0,
  rightmargin=0,
  roundcorner=5,
  needspace=0pt,
]{warning_mdfboxmdframed}

\newenvironment{warning_mdfboxadmon}[1][]{
\begin{warning_mdfboxmdframed}[frametitle=#1]
}
{
\end{warning_mdfboxmdframed}
}

% Admonition style "mdfbox" is an oval colored box based on mdframed
% "question" admon
\definecolor{mdfbox_question_background}{rgb}{1,1,1}
\newmdenv[
  skipabove=15pt,
  skipbelow=15pt,
  outerlinewidth=0,
  backgroundcolor=mdfbox_question_background,
  linecolor=black,
  linewidth=2pt,       % frame thickness
  frametitlebackgroundcolor=mdfbox_question_background,
  frametitlerule=true,
  frametitlefont=\normalfont\bfseries,
  shadow=false,        % frame shadow?
  shadowsize=11pt,
  leftmargin=0,
  rightmargin=0,
  roundcorner=5,
  needspace=0pt,
]{question_mdfboxmdframed}

\newenvironment{question_mdfboxadmon}[1][]{
\begin{question_mdfboxmdframed}[frametitle=#1]
}
{
\end{question_mdfboxmdframed}
}

% Admonition style "mdfbox" is an oval colored box based on mdframed
% "block" admon
\definecolor{mdfbox_block_background}{rgb}{1,1,1}
\newmdenv[
  skipabove=15pt,
  skipbelow=15pt,
  outerlinewidth=0,
  backgroundcolor=mdfbox_block_background,
  linecolor=black,
  linewidth=2pt,       % frame thickness
  frametitlebackgroundcolor=mdfbox_block_background,
  frametitlerule=true,
  frametitlefont=\normalfont\bfseries,
  shadow=false,        % frame shadow?
  shadowsize=11pt,
  leftmargin=0,
  rightmargin=0,
  roundcorner=5,
  needspace=0pt,
]{block_mdfboxmdframed}

\newenvironment{block_mdfboxadmon}[1][]{
\begin{block_mdfboxmdframed}[frametitle=#1]
}
{
\end{block_mdfboxmdframed}
}

% --- end of definitions of admonition environments ---

% prevent orhpans and widows
\clubpenalty = 10000
\widowpenalty = 10000

% --- end of standard preamble for documents ---


\usepackage[swedish]{babel}

\raggedbottom
\makeindex
\usepackage[totoc]{idxlayout}   % for index in the toc
\usepackage[nottoc]{tocbibind}  % for references/bibliography in the toc

%-------------------- end preamble ----------------------

\begin{document}

% matching end for #ifdef PREAMBLE
% #endif

\newcommand{\exercisesection}[1]{\subsection*{#1}}

\input{newcommands_keep}

% ------------------- main content ----------------------



% ----------------- title -------------------------

\thispagestyle{empty}

\begin{center}
{\LARGE\bf
\begin{spacing}{1.25}
Learning from data: Gaussian processes
\end{spacing}
}
\end{center}

% ----------------- author(s) -------------------------

\begin{center}
{\bf Christian Forssén}
\end{center}

    \begin{center}
% List of all institutions:
\centerline{{\small Department of Physics, Chalmers University of Technology, Sweden}}
\end{center}
    
% ----------------- end author(s) -------------------------

% --- begin date ---
\begin{center}
Oct 8, 2019
\end{center}
% --- end date ---

\vspace{1cm}


% !split
\subsection{Inference using Gaussian processes}

Assume that there is a set of input vectors with independent, predictor, variables
\[ \boldsymbol{X}_N \equiv \{ \boldsymbol{x}^{(n)}\}_{n=1}^N \]
and a set of target values
\[ \boldsymbol{t}_N \equiv \{ t^{(n)}\}_{n=1}^N. \]

\begin{itemize}
\item Note that we will use the symbol $t$ to denote the target, or response, variables in the context of Gaussian Processes. 

\item Furthermore, we will use the subscript $N$ to denote a vector of $N$ vectors (or scalars): $\boldsymbol{X}_N$ ($\boldsymbol{t}_N$)

\item While a single instance $i$ is denoted by a superscript: $\boldsymbol{x}^{(i)}$ ($t^{(i)}$).
\end{itemize}

\noindent
% !split
We will consider two different \emph{inference problems}:

\begin{enumerate}
\item The prediction of \emph{new target} $t^{(N+1)}$ given a new input $\boldsymbol{x}^{(N+1)}$

\item The inference of a \emph{function} $y(\boldsymbol{x})$ from the data.
\end{enumerate}

\noindent
% !split
The former can be expressed with the pdf
\[ 
p\left( t^{(N+1)} | \boldsymbol{t}_N, \boldsymbol{X}_{N}, \boldsymbol{x}^{(N+1)} \right)
\]
while the latter can be written using Bayes' formula (in these notes we will not be including information $I$ explicitly in the conditional probabilities)
\[ p\left( y(\boldsymbol{x}) | \boldsymbol{t}_N, \boldsymbol{X}_N \right)
= \frac{p\left( \boldsymbol{t}_N | y(\boldsymbol{x}), \boldsymbol{X}_N \right) p \left( y(\boldsymbol{x}) \right) }
{p\left( \boldsymbol{t}_N | \boldsymbol{X}_N \right) } \]

% !split
The inference of a function will obviously also allow to make predictions for new targets. 
However, we will need to consider in particular the second term in the numerator, which is the \textbf{prior} distribution on functions assumed in the model.

\begin{itemize}
\item This prior is implicit in parametric models with priors on the parameters.

\item The idea of Gaussian process modeling is to put a prior directly on the \textbf{space of functions} without parameterizing $y(\boldsymbol{x})$.

\item A Gaussian process can be thought of as a generalization of a Gaussian distribution over a finite vector space to a \textbf{function space of infinite dimension}.

\item Just as a Gaussian distribution is specified by its mean and covariance matrix, a Gaussian process is specified by a \textbf{mean and covariance function}.
\end{itemize}

\noindent
% !split

\begin{notice_mdfboxadmon}[Gaussian process]
A Gaussian process is a stochastic process (a collection of random variables indexed by time or space), such that every finite collection of those random variables has a multivariate normal distribution
\end{notice_mdfboxadmon} % title: Gaussian process



% !split
\paragraph{References:}
\begin{enumerate}
\item \href{{http://www.gaussianprocess.org/gpml}}{Gaussian Processes for Machine Learning}, Carl Edward Rasmussen and Chris Williams, the MIT Press, 2006, \href{{http://www.gaussianprocess.org/gpml/chapters}}{online version}.

\item \href{{https://sheffieldml.github.io/GPy/}}{GPy}: a Gaussian Process (GP) framework written in python, from the Sheffield machine learning group.
\end{enumerate}

\noindent
% !split
\subsection{Parametric approach}

Let us express $y(\boldsymbol{x})$ in terms of a model function $y(\boldsymbol{x}; \boldsymbol{\theta})$ that depends on a vector of model parameters $\boldsymbol{\theta}$.

For example, using a set of basis functions $\left\{ \phi^{(h)} (\boldsymbol{x}) \right\}_{h=1}^H$ with linear weights $\boldsymbol{\theta}_H$ we have
\[
y (\boldsymbol{x}, \boldsymbol{\theta}) = \sum_{h=1}^H \theta^{(h)} \phi^{(h)} (\boldsymbol{x})
\]


\begin{notice_mdfboxadmon}[Notice]
The basis functions can be non-linear such as Gaussians (aka \emph{radial basis functions})
\[
\phi^{(h)} (\boldsymbol{x}) = \exp \left[ -\frac{\left( \boldsymbol{x} - \boldsymbol{c}^{(h)} \right)^2}{2 (\sigma^{(h)})^2} \right].
\]

Still, this constitutes a linear model since $y (\boldsymbol{x}, \boldsymbol{\theta})$ depends linearly on the parameters $\boldsymbol{\theta}$.
\end{notice_mdfboxadmon} % title: Notice



The inference of model parameters should be a well-known problem by now. We state it in terms of Bayes theorem
\[
p \left( \boldsymbol{\theta} | \boldsymbol{t}_N, \boldsymbol{X}_N \right)
= \frac{ p \left( \boldsymbol{t}_N | \boldsymbol{\theta}, \boldsymbol{X}_N \right) p \left( \boldsymbol{\theta} \right)}{p \left( \boldsymbol{t}_N | \boldsymbol{X}_N \right)}
\]

Having solved this inference problem (e.g.~by linear regression) a prediction can be made through marginalization
\[
p\left( t^{(N+1)} | \boldsymbol{t}_N, \boldsymbol{X}_{N}, \boldsymbol{x}^{(N+1)} \right) 
= \int d^H \boldsymbol{\theta} 
p\left( t^{(N+1)} | \boldsymbol{\theta}, \boldsymbol{x}^{(N+1)} \right)
p \left( \boldsymbol{\theta} | \boldsymbol{t}_N, \boldsymbol{X}_N \right).
\]
Here it is important to note that the final answer does not make any explicit reference to our parametric representation of the unknown function $y(\boldsymbol{x})$.

Assuming that we have a fixed set of basis functions and Gaussian prior distributions (with zero mean) on the weights $\boldsymbol{\theta}$ we will show that:

\begin{itemize}
\item The joint pdf of the observed data given the model $p( \boldsymbol{t}_N |  \boldsymbol{X}_N)$, is a multivariate Gaussian with mean zero and with a covariance matrix that is determined by the basis functions.

\item This implies that the conditional distribution $p( t^{(N+1)} | \boldsymbol{t}_N, \boldsymbol{X}_{N+1})$, is also a multivariate Gaussian whose mean depends linearly on $\boldsymbol{t}_N$.
\end{itemize}

\noindent
\paragraph{Proof.}

\begin{notice_mdfboxadmon}[Sum of normally distributed random variables]
If $X$ and $Y$ are independent random variables that are normally distributed (and therefore also jointly so), then their sum is also normally distributed. i.e., $Z=X+Y$ is normally distributed with its mean being the sum of the two means, and its variance being the sum of the two variances.
\end{notice_mdfboxadmon} % title: Sum of normally distributed random variables



Consider the linear model and define the $N \times H$ design matrix $\boldsymbol{R}$ with elements
\[
R_{nh} \equiv \phi^{(h)} \left( \boldsymbol{x}^{(n)} \right).
\]

Then $\boldsymbol{y}_N = \boldsymbol{R} \boldsymbol{\theta}$ is the vector of model predictions, i.e.
\[
y^{(n)} = \sum_{h=1}^H R_{nh} \boldsymbol{\theta^{(h)}}.
\]

Assume that we have a Gaussian prior for the linear model weights $\boldsymbol{\theta}$ with zero mean and a diagonal covariance matrix
\[
p(\boldsymbol{\theta}) = \mathcal{N} \left( \boldsymbol{\theta}; 0, \sigma_\theta^2 \boldsymbol{I} \right).
\]

Now, since $y$ is a linear function of $\boldsymbol{\theta}$, it is also Gaussian distributed with mean zero. Its covariance matrix becomes
\[
\boldsymbol{Q} = \langle \boldsymbol{y} \boldsymbol{y}^T \rangle = \langle \boldsymbol{R} \boldsymbol{\theta} \boldsymbol{\theta}^T \boldsymbol{R}^T \rangle
= \sigma_\theta^2 \boldsymbol{R} \boldsymbol{R}^T,
\]
which implies that
\[
p(\boldsymbol{y}) = \mathcal{N} \left( \boldsymbol{y}; 0, \sigma_\theta^2 \boldsymbol{R} \boldsymbol{R}^T \right).
\]
This will be true for any set of points $\boldsymbol{X}_N$; which is the defining property of a \textbf{Gaussian process}.

\begin{itemize}
\item What about the target values $\boldsymbol{t}$?
\end{itemize}

\noindent
Well, if $t^{(n)}$ is assumed to differ by additive Gaussian noise, i.e., 
\[
t^{(n)} = y^{(n)} + \varepsilon^{(n)}, 
\]
where $\varepsilon^{(n)} \sim \mathcal{N} \left( 0, \sigma_\nu^2 \right)$; then $\boldsymbol{t}$ also has a Gaussian prior distribution
\[
p(\boldsymbol{t}) = \mathcal{N} \left( \boldsymbol{t}; 0, \boldsymbol{C} \right),
\]
where the covariance matrix of this target distribution is given by
\[
\boldsymbol{C} = \boldsymbol{Q} + \sigma_\nu^2 \boldsymbol{I} = \sigma_\theta^2 \boldsymbol{R} \boldsymbol{R}^T + \sigma_\nu^2 \boldsymbol{I}.
\]

% !split
\paragraph{The covariance matrix as the central object.}
The covariance matrices are given by
\[
Q_{nn'} = \sigma_\theta^2 \sum_h \phi^{(h)} \left( \boldsymbol{x}^{(n)} \right) \phi^{(h)} \left( \boldsymbol{x}^{(n')} \right),
\]
and
\[
C_{nn'} = Q_{nn'} + \delta_{nn'} \sigma_\nu^2.
\]
This means that the correlation between target values $t^{(n)}$ and $t^{(n')}$ is determined by the points $\boldsymbol{x}^{(n)}$, $\boldsymbol{x}^{(n')}$ and the behaviour of the basis functions.

% !split
\subsection{Non-parametric approach: Mean and covariance functions}

In fact, we don't really need the basis functions and their parameters anymore. The influence of these appear only in the covariance matrix that describes the distribution of the targets, which is our key object. We can replace the parametric model altogether with a \textbf{covariance function} $C( \boldsymbol{x}, \boldsymbol{x}' )$ which generates the  elements of the covariance matrix
\[
Q_{nn'} = C \left( \boldsymbol{x}^{(n)}, \boldsymbol{x}^{(n')} \right),
\]
for any set of points $\left\{ \boldsymbol{x}^{(n)} \right\}_{n=1}^N$.

Note, however, that $\boldsymbol{Q}$ must be positive-definite. This constrains the set of valid covariance functions.

Once we have defined a covariance function, the covariance matrix for the target values will be given by
\[
C_{nn'} = C \left( \boldsymbol{x}^{(n)}, \boldsymbol{x}^{(n')} \right) + \sigma_\nu^2 \delta_{nn'}.
\]

A wide range of different covariance contributions can be \href{{https://en.wikipedia.org/wiki/Gaussian_process#Covariance_functions}}{constructed}. These standard covariance functions are typically parametrized with hyperparameters $\boldsymbol{\theta}$ so that 
\[
C_{nn'} = C \left( \boldsymbol{x}^{(n)}, \boldsymbol{x}^{(n')}, \boldsymbol{\theta} \right) + \delta_{nn'} \Delta \left( \boldsymbol{x}^{(n)};  \boldsymbol{\theta} \right),
\]
where $\Delta$ is usually included as a flexible noise model.

% !split
\paragraph{Stationary kernels.}
The most common types of covariance functions are stationary, or translationally invariant, which implies that 
\[
C \left( \boldsymbol{x}^{(n)}, \boldsymbol{x}^{(n')}, \boldsymbol{\theta} \right) = D \left( \boldsymbol{x} - \boldsymbol{x}'; \boldsymbol{\theta} \right),
\]
where the function $D$ is often referred to as a \emph{kernel}.

A very standard kernel is the RBF (also known as Exponentiated Quadratic or Gaussian kernel) which is differentiable infinitely many times (hence, very smooth),
\[ 
C_\mathrm{RBF}(\mathbf{x},\mathbf{x}'; \boldsymbol{\theta}) = \theta_0 + \theta_1 \exp \left[ -\frac{1}{2} \sum_{i=1}^I \frac{(x_{i} - x_{i}')^2}{r_i^2} \right] 
\]
where $I$ denotes the dimensionality of the input space. The hyperparameters are: $\{ \theta_0, \theta_1, \vec{r} \}$. Sometimes, a single correlation length $r$ is used.


% !split
\subsection{GP models for regression}
Let us return to the problem of predicting $t^{(N+1)}$ given $\boldsymbol{t}_N$. The independent variables $\boldsymbol{X}_{N+1}$ are also given, but will be omitted from the conditional pdfs below.

The joint density is
\[
p \left( t^{(N+1)}, \boldsymbol{t}_N \right) = p \left( t^{(N+1)} | \boldsymbol{t}_N \right) p \left( \boldsymbol{t}_N \right) 
\quad \Rightarrow \quad
p \left( t^{(N+1)} | \boldsymbol{t}_N \right) = \frac{p \left( t^{(N+1)}, \boldsymbol{t}_N \right)}{p \left( \boldsymbol{t}_N \right) }.
\]

Since both $p \left( t^{(N+1)}, \boldsymbol{t}_N \right)$ and $p \left( \boldsymbol{t}_N \right)$ are Gaussian distributions, then the conditional distribution, obtained by the ratio, must also be a Gaussian. Let us use the notation $\boldsymbol{C}_{N+1}$ for the $(N+1) \times (N+1)$ covariance matrix for $\boldsymbol{t}_{N+1} = \left( \boldsymbol{t}_N, t^{(N+1)} \right)$. This implies that
\[
p \left( t^{(N+1)} | \boldsymbol{t}_N \right) \propto \exp \left[ -\frac{1}{2} \left( \boldsymbol{t}_N, t^{(N+1)} \right) \boldsymbol{C}_{N+1}^{-1} 
\begin{pmatrix}
\boldsymbol{t}_N \\
t^{(N+1)}
\end{pmatrix}
\right]
\]


\begin{summary_mdfboxadmon}[Summary]
The prediction of the (Gaussian) pdf for $t^{(N+1)}$ requires an inversion of the covariance matrix $\boldsymbol{C}_{N+1}^{-1}$.
\end{summary_mdfboxadmon} % title: Summary



\paragraph{Elegant approach using linear algebra tricks.}
Let us split the $\boldsymbol{C}_{N+1}$ covariance matrix into four different blocks
\[
\boldsymbol{C}_{N+1} =
\begin{pmatrix}
\boldsymbol{C}_N & \boldsymbol{k} \\
\boldsymbol{k}^T & \kappa
\end{pmatrix},
\]
where $\boldsymbol{C}_N$ is the $N \times N$ covariance matrix (which depends on the positions $\boldsymbol{X}_N$), $\boldsymbol{k}$ is an $N \times 1$ vector (that describes the covariance of $\boldsymbol{X}_N$ with $\boldsymbol{x}^{(N+1)}$), while $\kappa$ is the single diagonal element obtained from $\boldsymbol{x}^{(N+1)}$.

We can use the partitioned inverse equations (Barnett, 1979) to rewrite $\boldsymbol{C}_{N+1}^{-1}$ in terms of $\boldsymbol{C}_{N}^{-1}$ and $\boldsymbol{C}_{N}$ as follows
\[
\boldsymbol{C}_{N+1}^{-1} =
\begin{pmatrix}
\boldsymbol{M}_N & \boldsymbol{m} \\
\boldsymbol{m}^T & \mu
\end{pmatrix},
\]
where
\begin{align*}
\mu &= \left( \kappa - \boldsymbol{k}^T \boldsymbol{C}_N^{-1} \boldsymbol{k} \right)^{-1} \\
\boldsymbol{m} &= -\mu \boldsymbol{C}_N^{-1} \boldsymbol{t}_N \\
\boldsymbol{M}_N &= \boldsymbol{C}_N^{-1} + \frac{1}{\mu} \boldsymbol{m} \boldsymbol{m}^T.
\end{align*}


\begin{question_mdfboxadmon}[Question]
What are the dimensions of the different blocks? Check that the answer.
\end{question_mdfboxadmon} % title: Question



This implies that we can make a prediction for the Gaussian pdf of $t^{(N+1)}$ (meaning that we predict its value with an associated uncertainty) for an $N^3$ computational cost (the inversion of an $N \times N$ matrix).


\begin{summary_mdfboxadmon}[Summary]
The prediction for $t^{(N+1)}$ is a Gaussian
\[
p \left( t^{(N+1)} | \boldsymbol{t}_N \right) = \frac{1}{Z} \exp
\left[
-\frac{\left( t^{(N+1)} - \hat{t}^{(N+1)} \right)^2}{2 \sigma_{\hat{t}_{N+1}}^2}
\right]
\]
with
\begin{align*}
\mathrm{mean:} & \quad \hat{t}^{(N+1)} = \boldsymbol{k}^T \boldsymbol{C}_N^{-1} \boldsymbol{t}_N \\
\mathrm{variance:} & \quad \sigma_{\hat{t}_{N+1}}^2 = \kappa - \boldsymbol{k}^T \boldsymbol{C}_N^{-1} \boldsymbol{k}.
\end{align*}
\end{summary_mdfboxadmon} % title: Summary




\paragraph{Optimizing the GP model hyperparameters.}
To be added.

\subsection{GP emulators}

To be added.

% ------------------- end of main content ---------------

% #ifdef PREAMBLE
\end{document}
% #endif

