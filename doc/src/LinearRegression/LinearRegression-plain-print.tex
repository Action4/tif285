%%
%% Automatically generated file from DocOnce source
%% (https://github.com/hplgit/doconce/)
%%

% #define PREAMBLE

% #ifdef PREAMBLE
%-------------------- begin preamble ----------------------

\documentclass[%
oneside,                 % oneside: electronic viewing, twoside: printing
final,                   % draft: marks overfull hboxes, figures with paths
10pt]{article}

\listfiles               %  print all files needed to compile this document

\usepackage{relsize,makeidx,color,setspace,amsmath,amsfonts,amssymb}
\usepackage[table]{xcolor}
\usepackage{bm,ltablex,microtype}

\usepackage[pdftex]{graphicx}

\usepackage[T1]{fontenc}
%\usepackage[latin1]{inputenc}
\usepackage{ucs}
\usepackage[utf8x]{inputenc}

\usepackage{lmodern}         % Latin Modern fonts derived from Computer Modern

% Hyperlinks in PDF:
\definecolor{linkcolor}{rgb}{0,0,0.4}
\usepackage{hyperref}
\hypersetup{
    breaklinks=true,
    colorlinks=true,
    linkcolor=linkcolor,
    urlcolor=linkcolor,
    citecolor=black,
    filecolor=black,
    %filecolor=blue,
    pdfmenubar=true,
    pdftoolbar=true,
    bookmarksdepth=3   % Uncomment (and tweak) for PDF bookmarks with more levels than the TOC
    }
%\hyperbaseurl{}   % hyperlinks are relative to this root

\setcounter{tocdepth}{2}  % levels in table of contents

% --- fancyhdr package for fancy headers ---
\usepackage{fancyhdr}
\fancyhf{} % sets both header and footer to nothing
\renewcommand{\headrulewidth}{0pt}
\fancyfoot[LE,RO]{\thepage}
% Ensure copyright on titlepage (article style) and chapter pages (book style)
\fancypagestyle{plain}{
  \fancyhf{}
  \fancyfoot[C]{{\footnotesize \copyright\ 2018-2020, Christian Forssén. Released under CC Attribution-NonCommercial 4.0 license}}
%  \renewcommand{\footrulewidth}{0mm}
  \renewcommand{\headrulewidth}{0mm}
}
% Ensure copyright on titlepages with \thispagestyle{empty}
\fancypagestyle{empty}{
  \fancyhf{}
  \fancyfoot[C]{{\footnotesize \copyright\ 2018-2020, Christian Forssén. Released under CC Attribution-NonCommercial 4.0 license}}
  \renewcommand{\footrulewidth}{0mm}
  \renewcommand{\headrulewidth}{0mm}
}

\pagestyle{fancy}


\usepackage[framemethod=TikZ]{mdframed}

% --- begin definitions of admonition environments ---

% Admonition style "mdfbox" is an oval colored box based on mdframed
% "notice" admon
\definecolor{mdfbox_notice_background}{rgb}{1,1,1}
\newmdenv[
  skipabove=15pt,
  skipbelow=15pt,
  outerlinewidth=0,
  backgroundcolor=mdfbox_notice_background,
  linecolor=black,
  linewidth=2pt,       % frame thickness
  frametitlebackgroundcolor=mdfbox_notice_background,
  frametitlerule=true,
  frametitlefont=\normalfont\bfseries,
  shadow=false,        % frame shadow?
  shadowsize=11pt,
  leftmargin=0,
  rightmargin=0,
  roundcorner=5,
  needspace=0pt,
]{notice_mdfboxmdframed}

\newenvironment{notice_mdfboxadmon}[1][]{
\begin{notice_mdfboxmdframed}[frametitle=#1]
}
{
\end{notice_mdfboxmdframed}
}

% Admonition style "mdfbox" is an oval colored box based on mdframed
% "summary" admon
\definecolor{mdfbox_summary_background}{rgb}{1,1,1}
\newmdenv[
  skipabove=15pt,
  skipbelow=15pt,
  outerlinewidth=0,
  backgroundcolor=mdfbox_summary_background,
  linecolor=black,
  linewidth=2pt,       % frame thickness
  frametitlebackgroundcolor=mdfbox_summary_background,
  frametitlerule=true,
  frametitlefont=\normalfont\bfseries,
  shadow=false,        % frame shadow?
  shadowsize=11pt,
  leftmargin=0,
  rightmargin=0,
  roundcorner=5,
  needspace=0pt,
]{summary_mdfboxmdframed}

\newenvironment{summary_mdfboxadmon}[1][]{
\begin{summary_mdfboxmdframed}[frametitle=#1]
}
{
\end{summary_mdfboxmdframed}
}

% Admonition style "mdfbox" is an oval colored box based on mdframed
% "warning" admon
\definecolor{mdfbox_warning_background}{rgb}{1,1,1}
\newmdenv[
  skipabove=15pt,
  skipbelow=15pt,
  outerlinewidth=0,
  backgroundcolor=mdfbox_warning_background,
  linecolor=black,
  linewidth=2pt,       % frame thickness
  frametitlebackgroundcolor=mdfbox_warning_background,
  frametitlerule=true,
  frametitlefont=\normalfont\bfseries,
  shadow=false,        % frame shadow?
  shadowsize=11pt,
  leftmargin=0,
  rightmargin=0,
  roundcorner=5,
  needspace=0pt,
]{warning_mdfboxmdframed}

\newenvironment{warning_mdfboxadmon}[1][]{
\begin{warning_mdfboxmdframed}[frametitle=#1]
}
{
\end{warning_mdfboxmdframed}
}

% Admonition style "mdfbox" is an oval colored box based on mdframed
% "question" admon
\definecolor{mdfbox_question_background}{rgb}{1,1,1}
\newmdenv[
  skipabove=15pt,
  skipbelow=15pt,
  outerlinewidth=0,
  backgroundcolor=mdfbox_question_background,
  linecolor=black,
  linewidth=2pt,       % frame thickness
  frametitlebackgroundcolor=mdfbox_question_background,
  frametitlerule=true,
  frametitlefont=\normalfont\bfseries,
  shadow=false,        % frame shadow?
  shadowsize=11pt,
  leftmargin=0,
  rightmargin=0,
  roundcorner=5,
  needspace=0pt,
]{question_mdfboxmdframed}

\newenvironment{question_mdfboxadmon}[1][]{
\begin{question_mdfboxmdframed}[frametitle=#1]
}
{
\end{question_mdfboxmdframed}
}

% Admonition style "mdfbox" is an oval colored box based on mdframed
% "block" admon
\definecolor{mdfbox_block_background}{rgb}{1,1,1}
\newmdenv[
  skipabove=15pt,
  skipbelow=15pt,
  outerlinewidth=0,
  backgroundcolor=mdfbox_block_background,
  linecolor=black,
  linewidth=2pt,       % frame thickness
  frametitlebackgroundcolor=mdfbox_block_background,
  frametitlerule=true,
  frametitlefont=\normalfont\bfseries,
  shadow=false,        % frame shadow?
  shadowsize=11pt,
  leftmargin=0,
  rightmargin=0,
  roundcorner=5,
  needspace=0pt,
]{block_mdfboxmdframed}

\newenvironment{block_mdfboxadmon}[1][]{
\begin{block_mdfboxmdframed}[frametitle=#1]
}
{
\end{block_mdfboxmdframed}
}

% --- end of definitions of admonition environments ---

% prevent orhpans and widows
\clubpenalty = 10000
\widowpenalty = 10000

% --- end of standard preamble for documents ---


\usepackage[swedish]{babel}

\raggedbottom
\makeindex
\usepackage[totoc]{idxlayout}   % for index in the toc
\usepackage[nottoc]{tocbibind}  % for references/bibliography in the toc

%-------------------- end preamble ----------------------

\begin{document}

% matching end for #ifdef PREAMBLE
% #endif

\newcommand{\exercisesection}[1]{\subsection*{#1}}

\input{newcommands_keep}

% ------------------- main content ----------------------



% ----------------- title -------------------------

\thispagestyle{empty}

\begin{center}
{\LARGE\bf
\begin{spacing}{1.25}
Learning from data: Linear Regression
\end{spacing}
}
\end{center}

% ----------------- author(s) -------------------------

\begin{center}
{\bf Christian Forssén${}^{1}$} \\ [0mm]
\end{center}


\begin{center}
{\bf Morten Hjorth-Jensen${}^{2, 3}$} \\ [0mm]
\end{center}

\begin{center}
% List of all institutions:
\centerline{{\small ${}^1$Department of Physics, Chalmers University of Technology, Sweden}}
\centerline{{\small ${}^2$Department of Physics, University of Oslo}}
\centerline{{\small ${}^3$Department of Physics and Astronomy and National Superconducting Cyclotron Laboratory, Michigan State University}}
\end{center}
    
% ----------------- end author(s) -------------------------

% --- begin date ---
\begin{center}
Sep 1, 2020
\end{center}
% --- end date ---

\vspace{1cm}


% !split
\section{Linear regression}

% !split
\subsection{Why Linear Regression (aka Ordinary Least Squares)}

Fitting a continuous function with linear parameterization in terms of the parameters  $\bm{\theta}$.
\begin{itemize}
\item Often used for fitting a continuous function!

\item Gives an excellent introduction to central Machine Learning features with \textbf{understandable pedagogical} links to other methods like \textbf{Neural Networks}, \textbf{Support Vector Machines} etc

\item Analytical expression for the fitting parameters $\bm{\theta}$

\item Analytical expressions for statistical propertiers like mean values, variances, confidence intervals and more

\item Analytical relation with probabilistic interpretations 

\item Easy to introduce basic concepts like bias-variance tradeoff, cross-validation, resampling and regularization techniques and many other ML topics

\item Easy to code! And links well with classification problems and logistic regression and neural networks

\item Allows for \textbf{easy} hands-on understanding of gradient descent methods
\end{itemize}

\noindent
% !split
\paragraph{Regression analysis, overarching aims.}

\begin{block_mdfboxadmon}[]

Regression modeling deals with the description of a \textbf{response} variable(s) $y$ and how it varies as function of some \textbf{predictor} variable(s) $x$. The first variable is also often called the \textbf{dependent}, or the \textbf{outcome} variable while the second one can be called the \textbf{independent} variable, or the \textbf{explanatory} variable. Note also that each of these might be a vector of variables, meaning that there could be more than one response variable and more than one predictor variable. 

In general we will try to find a model $M$ that corresponds to a function $f_\theta(x)$ such that 
\[
y \approx f_\theta(x)
\]
In \textbf{linear regression} the dependence on the model parameters is \textbf{linear}, and this fact will make it possible to find an analytical expression for the optimal set of model parameters (as we will see below).
\end{block_mdfboxadmon} % title: 



% !split


\begin{block_mdfboxadmon}[]
When performing a regression analysis we will have access to a set of data $\mathcal{D}$ that consists of:
\begin{itemize}
\item $n$ cases $i = 0, 1, 2, \dots, n-1$ 
\end{itemize}

\noindent
For each case there is a
\begin{itemize}
\item (vector of) response variable(s) $y_i$ (observations);

\item (vector of) independent variable(s) $x_i$.
\end{itemize}

\noindent
Below, we will use boldface to denote the set of data, i.e., $\bm{y} = (y_0, y_1,\ldots, y_{n-1})$ and $\bm{x} = (x_0, x_1,\ldots, x_{n-1})$.

The independent variables can be turned into a number of \textbf{features}, and the key to a successful regression analysis is to identify the most relevant features. In physics, these would correspond to a set of \textbf{basis functions}.

Assume that there are $p$ features and we will use the (possibly confusing) notation
\begin{itemize}
\item $x_i=[x_{i0}, x_{i1}, \dots, x_{ip-1}]$ and from now on let $x$ denote the vector of features. See below for more explicit examples. 
\end{itemize}

\noindent
As our model will (in general) not predict the observations perfectly, we will write the relationship as
\[
y_i = f_\theta(x_i) + \epsilon_i,
\]
where $\epsilon_i$ is the error (or the \textbf{residual}).
\end{block_mdfboxadmon} % title: 



% !split


\begin{block_mdfboxadmon}[]
A regression analysis aims at finding the model parameters $\theta$ of a specified model $M$ such that the vector of errors $\bm{\epsilon}$ is minimized. You might ask the very relevant question what is specifically meant by minimizing a vector, and you will find that this is often achieved by minimizing a \textbf{cost} function that has been introduced without much motivation. This function might also be called a \textbf{loss} function or an \textbf{objective} function.

Alternatively, we could introduce the likelihood function $p(\bm{y} \vert \bm{x}, M(\theta))$.  It is the conditional distribution for the probability of making the observations $\bm{y}$ given the independent variable $\bm{x}$ and a model $M$, where $\bm{y}$ and $\bm{x}$ are contained in our data set $\mathcal{D}$. The parameters $\theta$ that maximizes this likelihood function is then our optimal set. We will later discuss likelihood functions in much more detail.

Having access to this ``optimal'' model, we have extracted a relationship between $\bm{y}$ and $\bm{x}$ that we can exploit to infer causal dependencies, make predictions, and many other things.
\end{block_mdfboxadmon} % title: 



% !split

\begin{block_mdfboxadmon}[]
The $p$ explanatory variables for the $n$ cases in the data set are normally represented by a matrix $\mathbf{X}$. The matrix $\mathbf{X}$ is called the \emph{design
matrix}.
\end{block_mdfboxadmon} % title: 





% !split
\paragraph{Example: Liquid-drop model for nuclear binding energies.}

\begin{block_mdfboxadmon}[]
In order to understand the relation among the predictors $p$, the set of data $\mathcal{D}_n$ and the target (outcome, output etc) $\bm{y}$,
consider the model we discussed for describing nuclear binding energies. 

There we assumed that we could parametrize the data using a polynomial approximation based on the liquid drop model.
Assuming 
\[
BE(A,N,Z) = a_0+a_1A+a_2A^{2/3}+a_3 Z^2 A^{-1/3}+a_4 (N-Z)^2 A^{-1},
\]
we have five features, that is the intercept (constant term, aka bias), the $A$ dependent term, the $A^{2/3}$ term and the $Z^2 A^{-1/3}$ and $(N-Z)^2 A^{-1}$ terms. Although the features are somewhat complicated functions of the independent variables $A,N,Z$, we note that the $p=5$ regression parameters $\theta = (a_0, a_1, a_2, a_3, a_4)$ enter linearly. Furthermore we have $n$ cases. It means that our design matrix is a 
$p\times n$ matrix $\bm{X}$.
\end{block_mdfboxadmon} % title: 




% !split
\subsection{Polynomial basis functions}

\begin{block_mdfboxadmon}[]
The perhaps simplest linear-regression approach is to assume we can parametrize our function in terms of a polynomial $f(x)$ of degree $p-1$. I.e.
\[
y(x_i)=f({x}_i)+\epsilon_i=\sum_{j=0}^{p-1} \theta_j x_i^j+\epsilon_i,
\]
where $\epsilon_i$ is the error in our approximation.
\end{block_mdfboxadmon} % title: 




% !split

\begin{block_mdfboxadmon}[]
For every set of values $y_i,x_i$ we have thus the corresponding set of equations
\begin{align*}
y_0&=\theta_0+\theta_1x_0^1+\theta_2x_0^2+\dots+\theta_{p-1}x_0^{p-1}+\epsilon_0\\
y_1&=\theta_0+\theta_1x_1^1+\theta_2x_1^2+\dots+\theta_{p-1}x_1^{p-1}+\epsilon_1\\
y_2&=\theta_0+\theta_1x_2^1+\theta_2x_2^2+\dots+\theta_{p-1}x_2^{p-1}+\epsilon_2\\
\dots & \dots \\
y_{n-1}&=\theta_0+\theta_1x_{n-1}^1+\theta_2x_{n-1}^2+\dots+\theta_{p-1}x_{n-1}^{p-1}+\epsilon_{n-1}.\\
\end{align*}
\end{block_mdfboxadmon} % title: 




% !split

\begin{block_mdfboxadmon}[]
Defining the vectors
\[
\bm{y} = [y_0,y_1, y_2,\dots, y_{n-1}]^T,
\]
and
\[
\bm{\theta} = [\theta_0,\theta_1, \theta_2,\dots, \theta_{p-1}]^T,
\]
and
\[
\bm{\epsilon} = [\epsilon_0,\epsilon_1, \epsilon_2,\dots, \epsilon_{n-1}]^T,
\]
and the design matrix
\[
\bm{X}=
\begin{bmatrix} 
1& x_{0}^1 &x_{0}^2& \dots & \dots &x_{0}^{p-1}\\
1& x_{1}^1 &x_{1}^2& \dots & \dots &x_{1}^{p-1}\\
1& x_{2}^1 &x_{2}^2& \dots & \dots &x_{2}^{p-1}\\                      
\dots& \dots &\dots& \dots & \dots &\dots\\
1& x_{n-1}^1 &x_{n-1}^2& \dots & \dots &x_{n-1}^{p-1}\\
\end{bmatrix} 
\]
we can rewrite our equations as
\[
\bm{y} = \bm{X}\bm{\theta}+\bm{\epsilon}.
\]
The above design matrix is called a \href{{https://en.wikipedia.org/wiki/Vandermonde_matrix}}{Vandermonde matrix}.
\end{block_mdfboxadmon} % title: 




% !split
\paragraph{General basis functions.}

\begin{block_mdfboxadmon}[]

We are obviously not limited to the above polynomial expansions.  We
could replace the various powers of $x$ with elements of Fourier
series or instead of $x_i^j$ we could have $\cos{(j x_i)}$ or $\sin{(j
x_i)}$, or time series or other orthogonal functions.  For every set
of values $y_i,x_i$ we can then generalize the equations to

\begin{align*}
y_0&=\theta_0x_{00}+\theta_1x_{01}+\theta_2x_{02}+\dots+\theta_{p-1}x_{0p-1}+\epsilon_0\\
y_1&=\theta_0x_{10}+\theta_1x_{11}+\theta_2x_{12}+\dots+\theta_{p-1}x_{1p-1}+\epsilon_1\\
y_2&=\theta_0x_{20}+\theta_1x_{21}+\theta_2x_{22}+\dots+\theta_{p-1}x_{2p-1}+\epsilon_2\\
\dots & \dots \\
y_{i}&=\theta_0x_{i0}+\theta_1x_{i1}+\theta_2x_{i2}+\dots+\theta_{p-1}x_{ip-1}+\epsilon_i\\
\dots & \dots \\
y_{n-1}&=\theta_0x_{n-1,0}+\theta_1x_{n-1,2}+\theta_2x_{n-1,2}+\dots+\theta_{p-1}x_{n-1,p-1}+\epsilon_{n-1}.\\
\end{align*}
\end{block_mdfboxadmon} % title: 




% !split

\begin{block_mdfboxadmon}[]
We redefine in turn the matrix $\bm{X}$ as
\[
\bm{X}=
\begin{bmatrix} 
x_{00}& x_{01} &x_{02}& \dots & \dots &x_{0,p-1}\\
x_{10}& x_{11} &x_{12}& \dots & \dots &x_{1,p-1}\\
x_{20}& x_{21} &x_{22}& \dots & \dots &x_{2,p-1}\\                      
\dots& \dots &\dots& \dots & \dots &\dots\\
x_{n-1,0}& x_{n-1,1} &x_{n-1,2}& \dots & \dots &x_{n-1,p-1}\\
\end{bmatrix} 
\]
and without loss of generality we rewrite again our equations as
\[
\bm{y} = \bm{X}\bm{\theta}+\bm{\epsilon}.
\]
The left-hand side of this equation is kwown. The error vector $\bm{\epsilon}$ and the parameter vector $\bm{\theta}$ are unknown quantities. How can we obtain the optimal set of $\theta_i$ values?
\end{block_mdfboxadmon} % title: 




% !split

\begin{block_mdfboxadmon}[]
We have defined the matrix $\bm{X}$ via the equations
\begin{align*}
y_0&=\theta_0x_{00}+\theta_1x_{01}+\theta_2x_{02}+\dots+\theta_{p-1}x_{0p-1}+\epsilon_0\\
y_1&=\theta_0x_{10}+\theta_1x_{11}+\theta_2x_{12}+\dots+\theta_{p-1}x_{1p-1}+\epsilon_1\\
y_2&=\theta_0x_{20}+\theta_1x_{21}+\theta_2x_{22}+\dots+\theta_{p-1}x_{2p-1}+\epsilon_1\\
\dots & \dots \\
y_{i}&=\theta_0x_{i0}+\theta_1x_{i1}+\theta_2x_{i2}+\dots+\theta_{p-1}x_{ip-1}+\epsilon_1\\
\dots & \dots \\
y_{n-1}&=\theta_0x_{n-1,0}+\theta_1x_{n-1,2}+\theta_2x_{n-1,2}+\dots+\theta_{p-1}x_{n-1,p-1}+\epsilon_{n-1}.\\
\end{align*}

Note that the design matrix 
 $\bm{X}\in {\mathbb{R}}^{n\times p}$, with the predictors refering to the column numbers and the entries $n$ being the row elements.
\end{block_mdfboxadmon} % title: 





% !split

\begin{block_mdfboxadmon}[]
With the above we use the design matrix to define the approximation $\bm{\tilde{y}}$ via the unknown quantity $\bm{\theta}$ as
\[
\bm{\tilde{y}}= \bm{X}\bm{\theta},
\]
and in order to find the optimal parameters $\theta_i$ instead of solving the above linear algebra problem, we define a function which gives a measure of the spread between the values $y_i$ (which represent hopefully the exact values) and the parameterized values $\tilde{y}_i$, namely
\[
C(\bm{\theta})=\frac{1}{n}\sum_{i=0}^{n-1}\left(y_i-\tilde{y}_i\right)^2=\frac{1}{n}\left\{\left(\bm{y}-\bm{\tilde{y}}\right)^T\left(\bm{y}-\bm{\tilde{y}}\right)\right\},
\]
or using the matrix $\bm{X}$ and in a more compact matrix-vector notation as
\[
C(\bm{\theta})=\frac{1}{n}\left\{\left(\bm{y}-\bm{X}\bm{\theta}\right)^T\left(\bm{y}-\bm{X}\bm{\theta}\right)\right\}.
\]
This function is one possible way to define the so-called \textbf{cost function}.



It is also common to define
the cost function as

\[
C(\bm{\theta})=\frac{1}{2n}\sum_{i=0}^{n-1}\left(y_i-\tilde{y}_i\right)^2,
\]
since when taking the first derivative with respect to the unknown parameters $\theta$, the factor of $2$ cancels out.
\end{block_mdfboxadmon} % title: 




% !split

\begin{block_mdfboxadmon}[]

The function 
\[
C(\bm{\theta})=\frac{1}{n}\left\{\left(\bm{y}-\bm{X}\bm{\theta}\right)^T\left(\bm{y}-\bm{X}\bm{\theta}\right)\right\},
\]
can be linked to the variance of the quantity $y_i$ if we interpret the latter as the mean value. 
When linking (see the discussion below) with the maximum likelihood approach, we will indeed interpret $y_i$ as a mean value
\[
y_{i}=\langle y_i \rangle = \theta_0x_{i,0}+\theta_1x_{i,1}+\theta_2x_{i,2}+\dots+\theta_{n-1}x_{i,n-1}+\epsilon_i,
\]

where $\langle y_i \rangle$ is the mean value. Keep in mind also that
till now we have treated $y_i$ as the exact value. Normally, the
response (dependent or outcome) variable $y_i$ the outcome of a
numerical experiment or another type of experiment and is thus only an
approximation to the true value. It is then always accompanied by an
error estimate, often limited to a statistical error estimate. For now, we
will treat $y_i$ as our exact value for the response variable.

In order to find the parameters $\theta_i$ we will then minimize the spread of $C(\bm{\theta})$, that is we are going to solve the problem
\[
{\displaystyle \min_{\bm{\theta}\in
{\mathbb{R}}^{p}}}\frac{1}{n}\left\{\left(\bm{y}-\bm{X}\bm{\theta}\right)^T\left(\bm{y}-\bm{X}\bm{\theta}\right)\right\}.
\]
In practical terms it means we will require
\[
\frac{\partial C(\bm{\theta})}{\partial \theta_j} = \frac{\partial }{\partial \theta_j}\left[ \frac{1}{n}\sum_{i=0}^{n-1}\left(y_i-\theta_0x_{i,0}-\theta_1x_{i,1}-\theta_2x_{i,2}-\dots-\theta_{n-1}x_{i,n-1}\right)^2\right]=0, 
\]
which results in
\[
\frac{\partial C(\bm{\theta})}{\partial \theta_j} = -\frac{2}{n}\left[ \sum_{i=0}^{n-1}x_{ij}\left(y_i-\theta_0x_{i,0}-\theta_1x_{i,1}-\theta_2x_{i,2}-\dots-\theta_{n-1}x_{i,n-1}\right)\right]=0, 
\]
or in a matrix-vector form as
\[
\frac{\partial C(\bm{\theta})}{\partial \bm{\theta}} = 0 = \bm{X}^T\left( \bm{y}-\bm{X}\bm{\theta}\right).  
\]
\end{block_mdfboxadmon} % title: 




% !split

\begin{block_mdfboxadmon}[]
We can rewrite
\[
\frac{\partial C(\bm{\theta})}{\partial \bm{\theta}} = 0 = \bm{X}^T\left( \bm{y}-\bm{X}\bm{\theta}\right),  
\]
as
\[
\bm{X}^T\bm{y} = \bm{X}^T\bm{X}\bm{\theta},  
\]
and if the matrix $\bm{X}^T\bm{X}$ is invertible we have the solution
\[
\bm{\theta} =\left(\bm{X}^T\bm{X}\right)^{-1}\bm{X}^T\bm{y}.
\]

We note also that since our design matrix is defined as $\bm{X}\in
{\mathbb{R}}^{n\times p}$, the product $\bm{X}^T\bm{X} \in
{\mathbb{R}}^{p\times p}$.  In the liquid drop model example from the Intro lecture, we had $p=5$ ($p \ll n$) meaning that we end up with inverting a small
$5\times 5$ matrix. This is a rather common situation, in many cases we end up with low-dimensional
matrices to invert, which
allow for the usage of direct linear algebra methods such as \textbf{LU} decomposition or \textbf{Singular Value Decomposition} (SVD) for finding the inverse of the matrix
$\bm{X}^T\bm{X}$.
\end{block_mdfboxadmon} % title: 




\begin{block_mdfboxadmon}[]
\textbf{Small question}: What kind of problems can we expect when inverting the matrix  $\bm{X}^T\bm{X}$?
\end{block_mdfboxadmon} % title: 



% !split
\subsection{Training scores}

We can easily test our fit by computing various \textbf{training scores}. Several such measures are used in machine learning applications. First we have the \textbf{Mean-Squared Error} (MSE)
\[
\mathrm{MSE}(\bm{\theta}) = \frac{1}{n} \sum_{i=1}^n \left( y_{\mathrm{data},i} - y_{\mathrm{model},i}(\bm{\theta}) \right)^2,
\]
where we have $n$ training data and our model is a function of the parameter vector $\bm{\theta}$.

Furthermore, we have the \textbf{mean absolute error} (MAE) defined as.
\[
\mathrm{MAE}(\bm{\theta}) = \frac{1}{n} \sum_{i=1}^n \left| y_{\mathrm{data},i} - y_{\mathrm{model},i}(\bm{\theta}) \right|,
\]

And the $R2$ score, also known as \emph{coefficient of determination} is
\[
\mathrm{R2}(\bm{\theta}) = 1 - \frac{\sum_{i=1}^n \left( y_{\mathrm{data},i} - y_{\mathrm{model},i}(\bm{\theta}) \right)^2}{\sum_{i=1}^n \left( y_{\mathrm{data},i} - \bar{y}_\mathrm{model}(\bm{\theta}) \right)^2},
\]
where $\bar{y}_\mathrm{model}(\bm{\theta}) = \frac{1}{n} \sum_{i=1}^n y_{\mathrm{model},i} (\bm{\theta})$ is the mean of the model predictions.


% !split
\subsection{The $\chi^2$ function}

\begin{block_mdfboxadmon}[]

Normally, the response (dependent or outcome) variable $y_i$ is the
outcome of a numerical experiment or another type of experiment and is
thus only an approximation to the true value. It is then always
accompanied by an error estimate, often limited to a statistical error
estimate given by a \textbf{standard deviation}. 

Introducing the standard deviation $\sigma_i$ for each measurement
$y_i$ (assuming uncorrelated errors), we define the so called $\chi^2$ function as

\[
\chi^2(\bm{\theta})=\frac{1}{n}\sum_{i=0}^{n-1}\frac{\left(y_i-\tilde{y}_i\right)^2}{\sigma_i^2}=\frac{1}{n}\left\{\left(\bm{y}-\bm{\tilde{y}}\right)^T \bm{\Sigma}^{-1}\left(\bm{y}-\bm{\tilde{y}}\right)\right\},
\]
where the matrix $\bm{\Sigma}$ is a diagonal $n \times n$ matrix with $\sigma_i^2$ as matrix elements.
\end{block_mdfboxadmon} % title: 



% !split

\begin{block_mdfboxadmon}[]

In order to find the parameters $\theta_i$ we will then minimize the $\chi^2(\bm{\theta})$ function by requiring
\[
\frac{\partial \chi^2(\bm{\theta})}{\partial \theta_j} = \frac{\partial }{\partial \theta_j}\left[ \frac{1}{n}\sum_{i=0}^{n-1}\left(\frac{y_i-\theta_0x_{i,0}-\theta_1x_{i,1}-\theta_2x_{i,2}-\dots-\theta_{n-1}x_{i,n-1}}{\sigma_i}\right)^2\right]=0, 
\]
which results in
\[
\frac{\partial \chi^2(\bm{\theta})}{\partial \theta_j} = -\frac{2}{n}\left[ \sum_{i=0}^{n-1}\frac{x_{ij}}{\sigma_i}\left(\frac{y_i-\theta_0x_{i,0}-\theta_1x_{i,1}-\theta_2x_{i,2}-\dots-\theta_{n-1}x_{i,n-1}}{\sigma_i}\right)\right]=0, 
\]
or in a matrix-vector form as
\[
\frac{\partial \chi^2(\bm{\theta})}{\partial \bm{\theta}} = 0 = \bm{A}^T\left( \bm{b}-\bm{A}\bm{\theta}\right).  
\]
where we have defined the matrix $\bm{A} =\bm{X} \bm{\Sigma}^{-1/2}$ with matrix elements $a_{ij} = x_{ij}/\sigma_i$ and the vector $\bm{b}$ with elements $b_i = y_i/\sigma_i$.
\end{block_mdfboxadmon} % title: 



% !split

\begin{block_mdfboxadmon}[]

We can rewrite
\[
\frac{\partial \chi^2(\bm{\theta})}{\partial \bm{\theta}} = 0 = \bm{A}^T\left( \bm{b}-\bm{A}\bm{\theta}\right),  
\]
as
\[
\bm{A}^T\bm{b} = \bm{A}^T\bm{A}\bm{\theta},  
\]
and if the matrix $\bm{A}^T\bm{A}$ is invertible we have the solution
\[
\bm{\theta} =\left(\bm{A}^T\bm{A}\right)^{-1}\bm{A}^T\bm{b}.
\]
\end{block_mdfboxadmon} % title: 



% !split

\begin{block_mdfboxadmon}[]

If we then introduce the matrix
\[
\bm{H} =  \left(\bm{A}^T\bm{A}\right)^{-1},
\]
we have then the following expression for the parameters $\theta_j$ (the matrix elements of $\bm{H}$ are $h_{ij}$)
\[
\theta_j = \sum_{k=0}^{p-1}h_{jk}\sum_{i=0}^{n-1}\frac{y_i}{\sigma_i}\frac{x_{ik}}{\sigma_i} = \sum_{k=0}^{p-1}h_{jk}\sum_{i=0}^{n-1}b_ia_{ik}
\]
We state without proof the expression for the uncertainty  in the parameters $\theta_j$ as (we leave this as an exercise)
\[
\sigma^2(\theta_j) = \sum_{i=0}^{n-1}\sigma_i^2\left( \frac{\partial \theta_j}{\partial y_i}\right)^2, 
\]
resulting in 
\[
\sigma^2(\theta_j) = \left(\sum_{k=0}^{p-1}h_{jk}\sum_{i=0}^{n-1}a_{ik}\right)\left(\sum_{l=0}^{p-1}h_{jl}\sum_{m=0}^{n-1}a_{ml}\right) = h_{jj}!
\]
\end{block_mdfboxadmon} % title: 



% ------------------- end of main content ---------------

% #ifdef PREAMBLE
\end{document}
% #endif

