%%
%% Automatically generated file from DocOnce source
%% (https://github.com/hplgit/doconce/)
%%

% #define PREAMBLE

% #ifdef PREAMBLE
%-------------------- begin preamble ----------------------

\documentclass[%
oneside,                 % oneside: electronic viewing, twoside: printing
final,                   % draft: marks overfull hboxes, figures with paths
10pt]{article}

\listfiles               %  print all files needed to compile this document

\usepackage{relsize,makeidx,color,setspace,amsmath,amsfonts,amssymb}
\usepackage[table]{xcolor}
\usepackage{bm,ltablex,microtype}

\usepackage[pdftex]{graphicx}

\usepackage[T1]{fontenc}
%\usepackage[latin1]{inputenc}
\usepackage{ucs}
\usepackage[utf8x]{inputenc}

\usepackage{lmodern}         % Latin Modern fonts derived from Computer Modern

% Hyperlinks in PDF:
\definecolor{linkcolor}{rgb}{0,0,0.4}
\usepackage{hyperref}
\hypersetup{
    breaklinks=true,
    colorlinks=true,
    linkcolor=linkcolor,
    urlcolor=linkcolor,
    citecolor=black,
    filecolor=black,
    %filecolor=blue,
    pdfmenubar=true,
    pdftoolbar=true,
    bookmarksdepth=3   % Uncomment (and tweak) for PDF bookmarks with more levels than the TOC
    }
%\hyperbaseurl{}   % hyperlinks are relative to this root

\setcounter{tocdepth}{2}  % levels in table of contents

% --- fancyhdr package for fancy headers ---
\usepackage{fancyhdr}
\fancyhf{} % sets both header and footer to nothing
\renewcommand{\headrulewidth}{0pt}
\fancyfoot[LE,RO]{\thepage}
% Ensure copyright on titlepage (article style) and chapter pages (book style)
\fancypagestyle{plain}{
  \fancyhf{}
  \fancyfoot[C]{{\footnotesize \copyright\ 2018-2019, Christian Forssén. Released under CC Attribution-NonCommercial 4.0 license}}
%  \renewcommand{\footrulewidth}{0mm}
  \renewcommand{\headrulewidth}{0mm}
}
% Ensure copyright on titlepages with \thispagestyle{empty}
\fancypagestyle{empty}{
  \fancyhf{}
  \fancyfoot[C]{{\footnotesize \copyright\ 2018-2019, Christian Forssén. Released under CC Attribution-NonCommercial 4.0 license}}
  \renewcommand{\footrulewidth}{0mm}
  \renewcommand{\headrulewidth}{0mm}
}

\pagestyle{fancy}


\usepackage[framemethod=TikZ]{mdframed}

% --- begin definitions of admonition environments ---

% Admonition style "mdfbox" is an oval colored box based on mdframed
% "notice" admon
\definecolor{mdfbox_notice_background}{rgb}{1,1,1}
\newmdenv[
  skipabove=15pt,
  skipbelow=15pt,
  outerlinewidth=0,
  backgroundcolor=mdfbox_notice_background,
  linecolor=black,
  linewidth=2pt,       % frame thickness
  frametitlebackgroundcolor=mdfbox_notice_background,
  frametitlerule=true,
  frametitlefont=\normalfont\bfseries,
  shadow=false,        % frame shadow?
  shadowsize=11pt,
  leftmargin=0,
  rightmargin=0,
  roundcorner=5,
  needspace=0pt,
]{notice_mdfboxmdframed}

\newenvironment{notice_mdfboxadmon}[1][]{
\begin{notice_mdfboxmdframed}[frametitle=#1]
}
{
\end{notice_mdfboxmdframed}
}

% Admonition style "mdfbox" is an oval colored box based on mdframed
% "summary" admon
\definecolor{mdfbox_summary_background}{rgb}{1,1,1}
\newmdenv[
  skipabove=15pt,
  skipbelow=15pt,
  outerlinewidth=0,
  backgroundcolor=mdfbox_summary_background,
  linecolor=black,
  linewidth=2pt,       % frame thickness
  frametitlebackgroundcolor=mdfbox_summary_background,
  frametitlerule=true,
  frametitlefont=\normalfont\bfseries,
  shadow=false,        % frame shadow?
  shadowsize=11pt,
  leftmargin=0,
  rightmargin=0,
  roundcorner=5,
  needspace=0pt,
]{summary_mdfboxmdframed}

\newenvironment{summary_mdfboxadmon}[1][]{
\begin{summary_mdfboxmdframed}[frametitle=#1]
}
{
\end{summary_mdfboxmdframed}
}

% Admonition style "mdfbox" is an oval colored box based on mdframed
% "warning" admon
\definecolor{mdfbox_warning_background}{rgb}{1,1,1}
\newmdenv[
  skipabove=15pt,
  skipbelow=15pt,
  outerlinewidth=0,
  backgroundcolor=mdfbox_warning_background,
  linecolor=black,
  linewidth=2pt,       % frame thickness
  frametitlebackgroundcolor=mdfbox_warning_background,
  frametitlerule=true,
  frametitlefont=\normalfont\bfseries,
  shadow=false,        % frame shadow?
  shadowsize=11pt,
  leftmargin=0,
  rightmargin=0,
  roundcorner=5,
  needspace=0pt,
]{warning_mdfboxmdframed}

\newenvironment{warning_mdfboxadmon}[1][]{
\begin{warning_mdfboxmdframed}[frametitle=#1]
}
{
\end{warning_mdfboxmdframed}
}

% Admonition style "mdfbox" is an oval colored box based on mdframed
% "question" admon
\definecolor{mdfbox_question_background}{rgb}{1,1,1}
\newmdenv[
  skipabove=15pt,
  skipbelow=15pt,
  outerlinewidth=0,
  backgroundcolor=mdfbox_question_background,
  linecolor=black,
  linewidth=2pt,       % frame thickness
  frametitlebackgroundcolor=mdfbox_question_background,
  frametitlerule=true,
  frametitlefont=\normalfont\bfseries,
  shadow=false,        % frame shadow?
  shadowsize=11pt,
  leftmargin=0,
  rightmargin=0,
  roundcorner=5,
  needspace=0pt,
]{question_mdfboxmdframed}

\newenvironment{question_mdfboxadmon}[1][]{
\begin{question_mdfboxmdframed}[frametitle=#1]
}
{
\end{question_mdfboxmdframed}
}

% Admonition style "mdfbox" is an oval colored box based on mdframed
% "block" admon
\definecolor{mdfbox_block_background}{rgb}{1,1,1}
\newmdenv[
  skipabove=15pt,
  skipbelow=15pt,
  outerlinewidth=0,
  backgroundcolor=mdfbox_block_background,
  linecolor=black,
  linewidth=2pt,       % frame thickness
  frametitlebackgroundcolor=mdfbox_block_background,
  frametitlerule=true,
  frametitlefont=\normalfont\bfseries,
  shadow=false,        % frame shadow?
  shadowsize=11pt,
  leftmargin=0,
  rightmargin=0,
  roundcorner=5,
  needspace=0pt,
]{block_mdfboxmdframed}

\newenvironment{block_mdfboxadmon}[1][]{
\begin{block_mdfboxmdframed}[frametitle=#1]
}
{
\end{block_mdfboxmdframed}
}

% --- end of definitions of admonition environments ---

% prevent orhpans and widows
\clubpenalty = 10000
\widowpenalty = 10000

% --- end of standard preamble for documents ---


\usepackage[swedish]{babel}

\raggedbottom
\makeindex
\usepackage[totoc]{idxlayout}   % for index in the toc
\usepackage[nottoc]{tocbibind}  % for references/bibliography in the toc

%-------------------- end preamble ----------------------

\begin{document}

% matching end for #ifdef PREAMBLE
% #endif

\newcommand{\exercisesection}[1]{\subsection*{#1}}

\input{newcommands_keep}

% ------------------- main content ----------------------



% ----------------- title -------------------------

\thispagestyle{empty}

\begin{center}
{\LARGE\bf
\begin{spacing}{1.25}
Learning from data: Linear Regression
\end{spacing}
}
\end{center}

% ----------------- author(s) -------------------------

\begin{center}
{\bf Christian Forssén${}^{1}$} \\ [0mm]
\end{center}


\begin{center}
{\bf Morten Hjorth-Jensen${}^{2, 3}$} \\ [0mm]
\end{center}

\begin{center}
% List of all institutions:
\centerline{{\small ${}^1$Department of Physics, Chalmers University of Technology, Sweden}}
\centerline{{\small ${}^2$Department of Physics, University of Oslo}}
\centerline{{\small ${}^3$Department of Physics and Astronomy and National Superconducting Cyclotron Laboratory, Michigan State University}}
\end{center}
    
% ----------------- end author(s) -------------------------

% --- begin date ---
\begin{center}
Sep 1, 2019
\end{center}
% --- end date ---

\vspace{1cm}


% !split
\subsection{Why Linear Regression (aka Ordinary Least Squares and family)}

Fitting a continuous function with linear parameterization in terms of the parameters  $\bm{\beta}$.
\begin{itemize}
\item Method of choice for fitting a continuous function!

\item Gives an excellent introduction to central Machine Learning features with \textbf{understandable pedagogical} links to other methods like \textbf{Neural Networks}, \textbf{Support Vector Machines} etc

\item Analytical expression for the fitting parameters $\bm{\beta}$

\item Analytical expressions for statistical propertiers like mean values, variances, confidence intervals and more

\item Analytical relation with probabilistic interpretations 

\item Easy to introduce basic concepts like bias-variance tradeoff, cross-validation, resampling and regularization techniques and many other ML topics

\item Easy to code! And links well with classification problems and logistic regression and neural networks

\item Allows for \textbf{easy} hands-on understanding of gradient descent methods

\item and many more features
\end{itemize}

\noindent
% !split
\paragraph{Regression analysis, overarching aims.}

\begin{block_mdfboxadmon}[]

Regression modeling deals with the description of  the sampling distribution of a given random variable $y$ and how it varies as function of another variable or a set of such variables $\bm{x} =[x_0, x_1,\dots, x_{n-1}]^T$. 
The first variable is called the \textbf{dependent}, the \textbf{outcome} or the \textbf{response} variable while the set of variables $\bm{x}$ is called the independent variable, or the predictor variable or the explanatory variable. 

A regression model aims at finding a likelihood function $p(\bm{y}\vert \bm{x})$, that is the conditional distribution for $\bm{y}$ with a given $\bm{x}$. The estimation of  $p(\bm{y}\vert \bm{x})$ is made using a data set with 
\begin{itemize}
\item $n$ cases $i = 0, 1, 2, \dots, n-1$ 

\item Response (target, dependent or outcome) variable $y_i$ with $i = 0, 1, 2, \dots, n-1$ 

\item $p$ so-called explanatory (independent or predictor) variables $\bm{x}_i=[x_{i0}, x_{i1}, \dots, x_{ip-1}]$ with $i = 0, 1, 2, \dots, n-1$ and explanatory variables running from $0$ to $p-1$. See below for more explicit examples.   
\end{itemize}

\noindent
 The goal of the regression analysis is to extract/exploit relationship between $\bm{y}$ and $\bm{x}$ in or to infer causal dependencies, approximations to the likelihood functions, functional relationships and to make predictions, making fits and many other things.
\end{block_mdfboxadmon} % title: 



% !split

\begin{block_mdfboxadmon}[]


Consider an experiment in which $p$ characteristics of $n$ samples are
measured. The data from this experiment, for various explanatory variables $p$ are normally represented by a matrix  
$\mathbf{X}$.

The matrix $\mathbf{X}$ is called the \emph{design
matrix}. Additional information of the samples is available in the
form of $\bm{y}$ (also as above). The variable $\bm{y}$ is
generally referred to as the \emph{response variable}. The aim of
regression analysis is to explain $\bm{y}$ in terms of
$\bm{X}$ through a functional relationship like $y_i =
f(\mathbf{X}_{i,\ast})$. When no prior knowledge on the form of
$f(\cdot)$ is available, it is common to assume a linear relationship
between $\bm{X}$ and $\bm{y}$. This assumption gives rise to
the \emph{linear regression model} where $\bm{\beta} = [\beta_0, \ldots,
\beta_{p-1}]^{T}$ are the \emph{regression parameters}. 

Linear regression gives us a set of analytical equations for the parameters $\beta_j$.
\end{block_mdfboxadmon} % title: 





% !split
\paragraph{Example: Liquid-drop model for nuclear binding energies.}

\begin{block_mdfboxadmon}[]
In order to understand the relation among the predictors $p$, the set of data $n$ and the target (outcome, output etc) $\bm{y}$,
consider the model we discussed for describing nuclear binding energies. 

There we assumed that we could parametrize the data using a polynomial approximation based on the liquid drop model.
Assuming 
\[
BE(A) = a_0+a_1A+a_2A^{2/3}+a_3A^{-1/3}+a_4A^{-1},
\]
we have five predictors, that is the intercept, the $A$ dependent term, the $A^{2/3}$ term and the $A^{-1/3}$ and $A^{-1}$ terms.
This gives $p=0,1,2,3,4$. Furthermore we have $n$ entries for each predictor. It means that our design matrix is a 
$p\times n$ matrix $\bm{X}$.
\end{block_mdfboxadmon} % title: 




% !split
\subsection{General linear models}
\paragraph{Polynomial basis functions.}

\begin{block_mdfboxadmon}[]
Before we proceed let us study a case from linear algebra where we aim at fitting a set of data $\bm{y}=[y_0,y_1,\dots,y_{n-1}]$. We could think of these data as a result of an experiment or a complicated numerical experiment. These data are functions of a series of variables $\bm{x}=[x_0,x_1,\dots,x_{n-1}]$, that is $y_i = y(x_i)$ with $i=0,1,2,\dots,n-1$. The variables $x_i$ could represent physical quantities like time, temperature, position etc. We assume that $y(x)$ is a smooth function. 

Since obtaining these data points may not be trivial, we want to use these data to fit a function which can allow us to make predictions for values of $y$ which are not in the present set. The perhaps simplest approach is to assume we can parametrize our function in terms of a polynomial of degree $n-1$ with $n$ points, that is
\[
y=y(x) \rightarrow y(x_i)=\tilde{y}_i+\epsilon_i=\sum_{j=0}^{n-1} \beta_j x_i^j+\epsilon_i,
\]
where $\epsilon_i$ is the error in our approximation.
\end{block_mdfboxadmon} % title: 




% !split

\begin{block_mdfboxadmon}[]
For every set of values $y_i,x_i$ we have thus the corresponding set of equations
\begin{align*}
y_0&=\beta_0+\beta_1x_0^1+\beta_2x_0^2+\dots+\beta_{n-1}x_0^{n-1}+\epsilon_0\\
y_1&=\beta_0+\beta_1x_1^1+\beta_2x_1^2+\dots+\beta_{n-1}x_1^{n-1}+\epsilon_1\\
y_2&=\beta_0+\beta_1x_2^1+\beta_2x_2^2+\dots+\beta_{n-1}x_2^{n-1}+\epsilon_2\\
\dots & \dots \\
y_{n-1}&=\beta_0+\beta_1x_{n-1}^1+\beta_2x_{n-1}^2+\dots+\beta_{n-1}x_{n-1}^{n-1}+\epsilon_{n-1}.\\
\end{align*}
\end{block_mdfboxadmon} % title: 




% !split

\begin{block_mdfboxadmon}[]
Defining the vectors
\[
\bm{y} = [y_0,y_1, y_2,\dots, y_{n-1}]^T,
\]
and
\[
\bm{\beta} = [\beta_0,\beta_1, \beta_2,\dots, \beta_{n-1}]^T,
\]
and
\[
\bm{\epsilon} = [\epsilon_0,\epsilon_1, \epsilon_2,\dots, \epsilon_{n-1}]^T,
\]
and the design matrix
\[
\bm{X}=
\begin{bmatrix} 
1& x_{0}^1 &x_{0}^2& \dots & \dots &x_{0}^{n-1}\\
1& x_{1}^1 &x_{1}^2& \dots & \dots &x_{1}^{n-1}\\
1& x_{2}^1 &x_{2}^2& \dots & \dots &x_{2}^{n-1}\\                      
\dots& \dots &\dots& \dots & \dots &\dots\\
1& x_{n-1}^1 &x_{n-1}^2& \dots & \dots &x_{n-1}^{n-1}\\
\end{bmatrix} 
\]
we can rewrite our equations as
\[
\bm{y} = \bm{X}\bm{\beta}+\bm{\epsilon}.
\]
The above design matrix is called a \href{{https://en.wikipedia.org/wiki/Vandermonde_matrix}}{Vandermonde matrix}.
\end{block_mdfboxadmon} % title: 




% !split
\paragraph{General basis functions.}

\begin{block_mdfboxadmon}[]

We are obviously not limited to the above polynomial expansions.  We
could replace the various powers of $x$ with elements of Fourier
series or instead of $x_i^j$ we could have $\cos{(j x_i)}$ or $\sin{(j
x_i)}$, or time series or other orthogonal functions.  For every set
of values $y_i,x_i$ we can then generalize the equations to

\begin{align*}
y_0&=\beta_0x_{00}+\beta_1x_{01}+\beta_2x_{02}+\dots+\beta_{n-1}x_{0n-1}+\epsilon_0\\
y_1&=\beta_0x_{10}+\beta_1x_{11}+\beta_2x_{12}+\dots+\beta_{n-1}x_{1n-1}+\epsilon_1\\
y_2&=\beta_0x_{20}+\beta_1x_{21}+\beta_2x_{22}+\dots+\beta_{n-1}x_{2n-1}+\epsilon_2\\
\dots & \dots \\
y_{i}&=\beta_0x_{i0}+\beta_1x_{i1}+\beta_2x_{i2}+\dots+\beta_{n-1}x_{in-1}+\epsilon_i\\
\dots & \dots \\
y_{n-1}&=\beta_0x_{n-1,0}+\beta_1x_{n-1,2}+\beta_2x_{n-1,2}+\dots+\beta_{n-1}x_{n-1,n-1}+\epsilon_{n-1}.\\
\end{align*}

\textbf{Note that we have $p=n$ here. The matrix is symmetric. This is generally not the case!}
\end{block_mdfboxadmon} % title: 




% !split

\begin{block_mdfboxadmon}[]
We redefine in turn the matrix $\bm{X}$ as
\[
\bm{X}=
\begin{bmatrix} 
x_{00}& x_{01} &x_{02}& \dots & \dots &x_{0,n-1}\\
x_{10}& x_{11} &x_{12}& \dots & \dots &x_{1,n-1}\\
x_{20}& x_{21} &x_{22}& \dots & \dots &x_{2,n-1}\\                      
\dots& \dots &\dots& \dots & \dots &\dots\\
x_{n-1,0}& x_{n-1,1} &x_{n-1,2}& \dots & \dots &x_{n-1,n-1}\\
\end{bmatrix} 
\]
and without loss of generality we rewrite again  our equations as
\[
\bm{y} = \bm{X}\bm{\beta}+\bm{\epsilon}.
\]
The left-hand side of this equation is kwown. Our error vector $\bm{\epsilon}$ and the parameter vector $\bm{\beta}$ are our unknow quantities. How can we obtain the optimal set of $\beta_i$ values?
\end{block_mdfboxadmon} % title: 




% !split

\begin{block_mdfboxadmon}[]
We have defined the matrix $\bm{X}$ via the equations
\begin{align*}
y_0&=\beta_0x_{00}+\beta_1x_{01}+\beta_2x_{02}+\dots+\beta_{n-1}x_{0n-1}+\epsilon_0\\
y_1&=\beta_0x_{10}+\beta_1x_{11}+\beta_2x_{12}+\dots+\beta_{n-1}x_{1n-1}+\epsilon_1\\
y_2&=\beta_0x_{20}+\beta_1x_{21}+\beta_2x_{22}+\dots+\beta_{n-1}x_{2n-1}+\epsilon_1\\
\dots & \dots \\
y_{i}&=\beta_0x_{i0}+\beta_1x_{i1}+\beta_2x_{i2}+\dots+\beta_{n-1}x_{in-1}+\epsilon_1\\
\dots & \dots \\
y_{n-1}&=\beta_0x_{n-1,0}+\beta_1x_{n-1,2}+\beta_2x_{n-1,2}+\dots+\beta_{n-1}x_{n-1,n-1}+\epsilon_{n-1}.\\
\end{align*}

As we noted above, we stayed with a system with the design matrix 
 $\bm{X}\in {\mathbb{R}}^{n\times n}$, that is we have $p=n$. For reasons to come later (algorithmic arguments) we will hereafter define 
our matrix as $\bm{X}\in {\mathbb{R}}^{n\times p}$, with the predictors refering to the column numbers and the entries $n$ being the row elements.
\end{block_mdfboxadmon} % title: 





% !split

\begin{block_mdfboxadmon}[]
With the above we use the design matrix to define the approximation $\bm{\tilde{y}}$ via the unknown quantity $\bm{\beta}$ as
\[
\bm{\tilde{y}}= \bm{X}\bm{\beta},
\]
and in order to find the optimal parameters $\beta_i$ instead of solving the above linear algebra problem, we define a function which gives a measure of the spread between the values $y_i$ (which represent hopefully the exact values) and the parameterized values $\tilde{y}_i$, namely
\[
C(\bm{\beta})=\frac{1}{n}\sum_{i=0}^{n-1}\left(y_i-\tilde{y}_i\right)^2=\frac{1}{n}\left\{\left(\bm{y}-\bm{\tilde{y}}\right)^T\left(\bm{y}-\bm{\tilde{y}}\right)\right\},
\]
or using the matrix $\bm{X}$ and in a more compact matrix-vector notation as
\[
C(\bm{\beta})=\frac{1}{n}\left\{\left(\bm{y}-\bm{X}^T\bm{\beta}\right)^T\left(\bm{y}-\bm{X}^T\bm{\beta}\right)\right\}.
\]
This function is one possible way to define the so-called cost function.



It is also common to define
the function $Q$ as

\[
C(\bm{\beta})=\frac{1}{2n}\sum_{i=0}^{n-1}\left(y_i-\tilde{y}_i\right)^2,
\]
since when taking the first derivative with respect to the unknown parameters $\beta$, the factor of $2$ cancels out.
\end{block_mdfboxadmon} % title: 




% !split

\begin{block_mdfboxadmon}[]

The function 
\[
C(\bm{\beta})=\frac{1}{n}\left\{\left(\bm{y}-\bm{X}\bm{\beta}\right)^T\left(\bm{y}-\bm{X}\bm{\beta}\right)\right\},
\]
can be linked to the variance of the quantity $y_i$ if we interpret the latter as the mean value. 
When linking (see the discussion below) with the maximum likelihood approach below, we will indeed interpret $y_i$ as a mean value
\[
y_{i}=\langle y_i \rangle = \beta_0x_{i,0}+\beta_1x_{i,1}+\beta_2x_{i,2}+\dots+\beta_{n-1}x_{i,n-1}+\epsilon_i,
\]

where $\langle y_i \rangle$ is the mean value. Keep in mind also that
till now we have treated $y_i$ as the exact value. Normally, the
response (dependent or outcome) variable $y_i$ the outcome of a
numerical experiment or another type of experiment and is thus only an
approximation to the true value. It is then always accompanied by an
error estimate, often limited to a statistical error estimate given by
the standard deviation discussed earlier. In the discussion here we
will treat $y_i$ as our exact value for the response variable.

In order to find the parameters $\beta_i$ we will then minimize the spread of $C(\bm{\beta})$, that is we are going to solve the problem
\[
{\displaystyle \min_{\bm{\beta}\in
{\mathbb{R}}^{p}}}\frac{1}{n}\left\{\left(\bm{y}-\bm{X}\bm{\beta}\right)^T\left(\bm{y}-\bm{X}\bm{\beta}\right)\right\}.
\]
In practical terms it means we will require
\[
\frac{\partial C(\bm{\beta})}{\partial \beta_j} = \frac{\partial }{\partial \beta_j}\left[ \frac{1}{n}\sum_{i=0}^{n-1}\left(y_i-\beta_0x_{i,0}-\beta_1x_{i,1}-\beta_2x_{i,2}-\dots-\beta_{n-1}x_{i,n-1}\right)^2\right]=0, 
\]
which results in
\[
\frac{\partial C(\bm{\beta})}{\partial \beta_j} = -\frac{2}{n}\left[ \sum_{i=0}^{n-1}x_{ij}\left(y_i-\beta_0x_{i,0}-\beta_1x_{i,1}-\beta_2x_{i,2}-\dots-\beta_{n-1}x_{i,n-1}\right)\right]=0, 
\]
or in a matrix-vector form as
\[
\frac{\partial C(\bm{\beta})}{\partial \bm{\beta}} = 0 = \bm{X}^T\left( \bm{y}-\bm{X}\bm{\beta}\right).  
\]
\end{block_mdfboxadmon} % title: 




% !split

\begin{block_mdfboxadmon}[]
We can rewrite
\[
\frac{\partial C(\bm{\beta})}{\partial \bm{\beta}} = 0 = \bm{X}^T\left( \bm{y}-\bm{X}\bm{\beta}\right),  
\]
as
\[
\bm{X}^T\bm{y} = \bm{X}^T\bm{X}\bm{\beta},  
\]
and if the matrix $\bm{X}^T\bm{X}$ is invertible we have the solution
\[
\bm{\beta} =\left(\bm{X}^T\bm{X}\right)^{-1}\bm{X}^T\bm{y}.
\]

We note also that since our design matrix is defined as $\bm{X}\in
{\mathbb{R}}^{n\times p}$, the product $\bm{X}^T\bm{X} \in
{\mathbb{R}}^{p\times p}$.  In the liquid drop model example from the Intro lecture, we had $p=5$ ($p \ll n$) meaning that we end up with inverting a small
$5\times 5$ matrix. This is a rather common situation, in many cases we end up with low-dimensional
matrices to invert. The methods discussed here and for many other
supervised learning algorithms like classification with logistic
regression or support vector machines, exhibit dimensionalities which
allow for the usage of direct linear algebra methods such as \textbf{LU} decomposition or \textbf{Singular Value Decomposition} (SVD) for finding the inverse of the matrix
$\bm{X}^T\bm{X}$.
\end{block_mdfboxadmon} % title: 




\begin{block_mdfboxadmon}[]
\textbf{Small question}: What kind of problems can we expect when inverting the matrix  $\bm{X}^T\bm{X}$?
\end{block_mdfboxadmon} % title: 



% !split
\paragraph{Some useful matrix and vector expressions.}
The following matrix and vector relation will be useful here and for the rest of the course. Vectors are always written as boldfaced lower case letters and 
matrices as upper case boldfaced letters. 

\[
\frac{\partial (\bm{b}^T\bm{a})}{\partial \bm{a}} = \bm{b},
\]
\[
\frac{\partial (\bm{a}^T\bm{A}\bm{a})}{\partial \bm{a}} = (\bm{A}+\bm{A}^T)\bm{a},
\]
\[
\frac{\partial tr(\bm{B}\bm{A})}{\partial \bm{A}} = \bm{B}^T,
\]
\[
\frac{\partial \log{\vert\bm{A}\vert}}{\partial \bm{A}} = (\bm{A}^{-1})^T.
\]

% !split

\begin{block_mdfboxadmon}[]
The residuals $\bm{\epsilon}$ are in turn given by
\[
\bm{\epsilon} = \bm{y}-\bm{\tilde{y}} = \bm{y}-\bm{X}\bm{\beta},
\]
and with 
\[
\bm{X}^T\left( \bm{y}-\bm{X}\bm{\beta}\right)= 0, 
\]
we have
\[
\bm{X}^T\bm{\epsilon}=\bm{X}^T\left( \bm{y}-\bm{X}\bm{\beta}\right)= 0, 
\]
meaning that the solution for $\bm{\beta}$ is the one which minimizes the residuals.  Later we will link this with the maximum likelihood approach.
\end{block_mdfboxadmon} % title: 



% !split
\subsection{Adding error analysis and training set up}

We can easily test our fit by computing various \textbf{cost functions} (or \textbf{training scores}). Several such cost functions are used in machine learning applications. First we have the \textbf{Mean-Squared Error} (MSE)
\[
\mathrm{MSE}(\bm{\beta}) = \frac{1}{n} \sum_{i=1}^n \left( y_{\mathrm{data},i} - y_{\mathrm{model},i}(\bm{\beta}) \right)^2,
\]
where we have $n$ training data and our model is a function of the parameter vector $\bm{\beta}$.

Furthermore, we have the \textbf{mean absolute error} (MAE) defined as.
\[
\mathrm{MAE}(\bm{\beta}) = \frac{1}{n} \sum_{i=1}^n \left| y_{\mathrm{data},i} - y_{\mathrm{model},i}(\bm{\beta}) \right|,
\]

And the $R2$ score, also known as \emph{coefficient of determination} is
\[
\mathrm{R2}(\bm{\beta}) = 1 - \frac{\sum_{i=1}^n \left( y_{\mathrm{data},i} - y_{\mathrm{model},i}(\bm{\beta}) \right)^2}{\sum_{i=1}^n \left( y_{\mathrm{data},i} - \bar{y}_\mathrm{model}(\bm{\beta}) \right)^2},
\]
where $\bar{y}_\mathrm{model}(\bm{\beta}) = \frac{1}{n} \sum_{i=1}^n y_{\mathrm{model},i} (\bm{\beta})$ is the mean of the model predictions.


% !split
\paragraph{The $\chi^2$ function.}

\begin{block_mdfboxadmon}[]

Normally, the response (dependent or outcome) variable $y_i$ is the
outcome of a numerical experiment or another type of experiment and is
thus only an approximation to the true value. It is then always
accompanied by an error estimate, often limited to a statistical error
estimate given by the standard deviation discussed earlier. In the
discussion here we will treat $y_i$ as our exact value for the
response variable.

Introducing the standard deviation $\sigma_i$ for each measurement
$y_i$, we define now the $\chi^2$ function (omitting the $1/n$ term)
as

\[
\chi^2(\bm{\beta})=\frac{1}{n}\sum_{i=0}^{n-1}\frac{\left(y_i-\tilde{y}_i\right)^2}{\sigma_i^2}=\frac{1}{n}\left\{\left(\bm{y}-\bm{\tilde{y}}\right)^T\frac{1}{\bm{\Sigma^2}}\left(\bm{y}-\bm{\tilde{y}}\right)\right\},
\]
where the matrix $\bm{\Sigma}$ is a diagonal matrix with $\sigma_i$ as matrix elements.
\end{block_mdfboxadmon} % title: 



% !split

\begin{block_mdfboxadmon}[]

In order to find the parameters $\beta_i$ we will then minimize the spread of $\chi^2(\bm{\beta})$ by requiring
\[
\frac{\partial \chi^2(\bm{\beta})}{\partial \beta_j} = \frac{\partial }{\partial \beta_j}\left[ \frac{1}{n}\sum_{i=0}^{n-1}\left(\frac{y_i-\beta_0x_{i,0}-\beta_1x_{i,1}-\beta_2x_{i,2}-\dots-\beta_{n-1}x_{i,n-1}}{\sigma_i}\right)^2\right]=0, 
\]
which results in
\[
\frac{\partial \chi^2(\bm{\beta})}{\partial \beta_j} = -\frac{2}{n}\left[ \sum_{i=0}^{n-1}\frac{x_{ij}}{\sigma_i}\left(\frac{y_i-\beta_0x_{i,0}-\beta_1x_{i,1}-\beta_2x_{i,2}-\dots-\beta_{n-1}x_{i,n-1}}{\sigma_i}\right)\right]=0, 
\]
or in a matrix-vector form as
\[
\frac{\partial \chi^2(\bm{\beta})}{\partial \bm{\beta}} = 0 = \bm{A}^T\left( \bm{b}-\bm{A}\bm{\beta}\right).  
\]
where we have defined the matrix $\bm{A} =\bm{X}/\bm{\Sigma}$ with matrix elements $a_{ij} = x_{ij}/\sigma_i$ and the vector $\bm{b}$ with elements $b_i = y_i/\sigma_i$.
\end{block_mdfboxadmon} % title: 



% !split

\begin{block_mdfboxadmon}[]

We can rewrite
\[
\frac{\partial \chi^2(\bm{\beta})}{\partial \bm{\beta}} = 0 = \bm{A}^T\left( \bm{b}-\bm{A}\bm{\beta}\right),  
\]
as
\[
\bm{A}^T\bm{b} = \bm{A}^T\bm{A}\bm{\beta},  
\]
and if the matrix $\bm{A}^T\bm{A}$ is invertible we have the solution
\[
\bm{\beta} =\left(\bm{A}^T\bm{A}\right)^{-1}\bm{A}^T\bm{b}.
\]
\end{block_mdfboxadmon} % title: 



% !split

\begin{block_mdfboxadmon}[]

If we then introduce the matrix
\[
\bm{H} =  \left(\bm{A}^T\bm{A}\right)^{-1},
\]
we have then the following expression for the parameters $\beta_j$ (the matrix elements of $\bm{H}$ are $h_{ij}$)
\[
\beta_j = \sum_{k=0}^{p-1}h_{jk}\sum_{i=0}^{n-1}\frac{y_i}{\sigma_i}\frac{x_{ik}}{\sigma_i} = \sum_{k=0}^{p-1}h_{jk}\sum_{i=0}^{n-1}b_ia_{ik}
\]
We state without proof the expression for the uncertainty  in the parameters $\beta_j$ as (we leave this as an exercise)
\[
\sigma^2(\beta_j) = \sum_{i=0}^{n-1}\sigma_i^2\left( \frac{\partial \beta_j}{\partial y_i}\right)^2, 
\]
resulting in 
\[
\sigma^2(\beta_j) = \left(\sum_{k=0}^{p-1}h_{jk}\sum_{i=0}^{n-1}a_{ik}\right)\left(\sum_{l=0}^{p-1}h_{jl}\sum_{m=0}^{n-1}a_{ml}\right) = h_{jj}!
\]
\end{block_mdfboxadmon} % title: 



% !split

\begin{block_mdfboxadmon}[]
The first step here is to approximate the function $y$ with a first-order polynomial, that is we write
\[
y=y(x) \rightarrow y(x_i) \approx \beta_0+\beta_1 x_i.
\]
By computing the derivatives of $\chi^2$ with respect to $\beta_0$ and $\beta_1$ show that these are given by
\[
\frac{\partial \chi^2(\bm{\beta})}{\partial \beta_0} = -2\left[ \frac{1}{n}\sum_{i=0}^{n-1}\left(\frac{y_i-\beta_0-\beta_1x_{i}}{\sigma_i^2}\right)\right]=0,
\]
and
\[
\frac{\partial \chi^2(\bm{\beta})}{\partial \beta_1} = -\frac{2}{n}\left[ \sum_{i=0}^{n-1}x_i\left(\frac{y_i-\beta_0-\beta_1x_{i}}{\sigma_i^2}\right)\right]=0.
\]
\end{block_mdfboxadmon} % title: 



% ------------------- end of main content ---------------

% #ifdef PREAMBLE
\end{document}
% #endif

