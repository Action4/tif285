%%
%% Automatically generated file from DocOnce source
%% (https://github.com/hplgit/doconce/)
%%

% #define PREAMBLE

% #ifdef PREAMBLE
%-------------------- begin preamble ----------------------

\documentclass[%
oneside,                 % oneside: electronic viewing, twoside: printing
final,                   % draft: marks overfull hboxes, figures with paths
10pt]{article}

\listfiles               %  print all files needed to compile this document

\usepackage{relsize,makeidx,color,setspace,amsmath,amsfonts,amssymb}
\usepackage[table]{xcolor}
\usepackage{bm,ltablex,microtype}

\usepackage[pdftex]{graphicx}

\usepackage[T1]{fontenc}
%\usepackage[latin1]{inputenc}
\usepackage{ucs}
\usepackage[utf8x]{inputenc}

\usepackage{lmodern}         % Latin Modern fonts derived from Computer Modern

% Hyperlinks in PDF:
\definecolor{linkcolor}{rgb}{0,0,0.4}
\usepackage{hyperref}
\hypersetup{
    breaklinks=true,
    colorlinks=true,
    linkcolor=linkcolor,
    urlcolor=linkcolor,
    citecolor=black,
    filecolor=black,
    %filecolor=blue,
    pdfmenubar=true,
    pdftoolbar=true,
    bookmarksdepth=3   % Uncomment (and tweak) for PDF bookmarks with more levels than the TOC
    }
%\hyperbaseurl{}   % hyperlinks are relative to this root

\setcounter{tocdepth}{2}  % levels in table of contents

% --- fancyhdr package for fancy headers ---
\usepackage{fancyhdr}
\fancyhf{} % sets both header and footer to nothing
\renewcommand{\headrulewidth}{0pt}
\fancyfoot[LE,RO]{\thepage}
% Ensure copyright on titlepage (article style) and chapter pages (book style)
\fancypagestyle{plain}{
  \fancyhf{}
  \fancyfoot[C]{{\footnotesize \copyright\ 2018-2020, Christian Forssén. Released under CC Attribution-NonCommercial 4.0 license}}
%  \renewcommand{\footrulewidth}{0mm}
  \renewcommand{\headrulewidth}{0mm}
}
% Ensure copyright on titlepages with \thispagestyle{empty}
\fancypagestyle{empty}{
  \fancyhf{}
  \fancyfoot[C]{{\footnotesize \copyright\ 2018-2020, Christian Forssén. Released under CC Attribution-NonCommercial 4.0 license}}
  \renewcommand{\footrulewidth}{0mm}
  \renewcommand{\headrulewidth}{0mm}
}

\pagestyle{fancy}


\usepackage[framemethod=TikZ]{mdframed}

% --- begin definitions of admonition environments ---

% Admonition style "mdfbox" is an oval colored box based on mdframed
% "notice" admon
\definecolor{mdfbox_notice_background}{rgb}{1,1,1}
\newmdenv[
  skipabove=15pt,
  skipbelow=15pt,
  outerlinewidth=0,
  backgroundcolor=mdfbox_notice_background,
  linecolor=black,
  linewidth=2pt,       % frame thickness
  frametitlebackgroundcolor=mdfbox_notice_background,
  frametitlerule=true,
  frametitlefont=\normalfont\bfseries,
  shadow=false,        % frame shadow?
  shadowsize=11pt,
  leftmargin=0,
  rightmargin=0,
  roundcorner=5,
  needspace=0pt,
]{notice_mdfboxmdframed}

\newenvironment{notice_mdfboxadmon}[1][]{
\begin{notice_mdfboxmdframed}[frametitle=#1]
}
{
\end{notice_mdfboxmdframed}
}

% Admonition style "mdfbox" is an oval colored box based on mdframed
% "summary" admon
\definecolor{mdfbox_summary_background}{rgb}{1,1,1}
\newmdenv[
  skipabove=15pt,
  skipbelow=15pt,
  outerlinewidth=0,
  backgroundcolor=mdfbox_summary_background,
  linecolor=black,
  linewidth=2pt,       % frame thickness
  frametitlebackgroundcolor=mdfbox_summary_background,
  frametitlerule=true,
  frametitlefont=\normalfont\bfseries,
  shadow=false,        % frame shadow?
  shadowsize=11pt,
  leftmargin=0,
  rightmargin=0,
  roundcorner=5,
  needspace=0pt,
]{summary_mdfboxmdframed}

\newenvironment{summary_mdfboxadmon}[1][]{
\begin{summary_mdfboxmdframed}[frametitle=#1]
}
{
\end{summary_mdfboxmdframed}
}

% Admonition style "mdfbox" is an oval colored box based on mdframed
% "warning" admon
\definecolor{mdfbox_warning_background}{rgb}{1,1,1}
\newmdenv[
  skipabove=15pt,
  skipbelow=15pt,
  outerlinewidth=0,
  backgroundcolor=mdfbox_warning_background,
  linecolor=black,
  linewidth=2pt,       % frame thickness
  frametitlebackgroundcolor=mdfbox_warning_background,
  frametitlerule=true,
  frametitlefont=\normalfont\bfseries,
  shadow=false,        % frame shadow?
  shadowsize=11pt,
  leftmargin=0,
  rightmargin=0,
  roundcorner=5,
  needspace=0pt,
]{warning_mdfboxmdframed}

\newenvironment{warning_mdfboxadmon}[1][]{
\begin{warning_mdfboxmdframed}[frametitle=#1]
}
{
\end{warning_mdfboxmdframed}
}

% Admonition style "mdfbox" is an oval colored box based on mdframed
% "question" admon
\definecolor{mdfbox_question_background}{rgb}{1,1,1}
\newmdenv[
  skipabove=15pt,
  skipbelow=15pt,
  outerlinewidth=0,
  backgroundcolor=mdfbox_question_background,
  linecolor=black,
  linewidth=2pt,       % frame thickness
  frametitlebackgroundcolor=mdfbox_question_background,
  frametitlerule=true,
  frametitlefont=\normalfont\bfseries,
  shadow=false,        % frame shadow?
  shadowsize=11pt,
  leftmargin=0,
  rightmargin=0,
  roundcorner=5,
  needspace=0pt,
]{question_mdfboxmdframed}

\newenvironment{question_mdfboxadmon}[1][]{
\begin{question_mdfboxmdframed}[frametitle=#1]
}
{
\end{question_mdfboxmdframed}
}

% Admonition style "mdfbox" is an oval colored box based on mdframed
% "block" admon
\definecolor{mdfbox_block_background}{rgb}{1,1,1}
\newmdenv[
  skipabove=15pt,
  skipbelow=15pt,
  outerlinewidth=0,
  backgroundcolor=mdfbox_block_background,
  linecolor=black,
  linewidth=2pt,       % frame thickness
  frametitlebackgroundcolor=mdfbox_block_background,
  frametitlerule=true,
  frametitlefont=\normalfont\bfseries,
  shadow=false,        % frame shadow?
  shadowsize=11pt,
  leftmargin=0,
  rightmargin=0,
  roundcorner=5,
  needspace=0pt,
]{block_mdfboxmdframed}

\newenvironment{block_mdfboxadmon}[1][]{
\begin{block_mdfboxmdframed}[frametitle=#1]
}
{
\end{block_mdfboxmdframed}
}

% --- end of definitions of admonition environments ---

% prevent orhpans and widows
\clubpenalty = 10000
\widowpenalty = 10000

% --- end of standard preamble for documents ---


\usepackage[swedish]{babel}

\raggedbottom
\makeindex
\usepackage[totoc]{idxlayout}   % for index in the toc
\usepackage[nottoc]{tocbibind}  % for references/bibliography in the toc

%-------------------- end preamble ----------------------

\begin{document}

% matching end for #ifdef PREAMBLE
% #endif

\newcommand{\exercisesection}[1]{\subsection*{#1}}

\input{newcommands_keep}

% ------------------- main content ----------------------



% ----------------- title -------------------------

\thispagestyle{empty}

\begin{center}
{\LARGE\bf
\begin{spacing}{1.25}
Learning from data: Neural networks, from the simple perceptron to deep learning
\end{spacing}
}
\end{center}

% ----------------- author(s) -------------------------

\begin{center}
{\bf Christian Forssén${}^{1}$} \\ [0mm]
\end{center}


\begin{center}
{\bf Morten Hjorth-Jensen${}^{2, 3}$} \\ [0mm]
\end{center}

\begin{center}
% List of all institutions:
\centerline{{\small ${}^1$Department of Physics, Chalmers University of Technology, Sweden}}
\centerline{{\small ${}^2$Department of Physics, University of Oslo}}
\centerline{{\small ${}^3$Department of Physics and Astronomy and National Superconducting Cyclotron Laboratory, Michigan State University}}
\end{center}
    
% ----------------- end author(s) -------------------------

% --- begin date ---
\begin{center}
Oct 19, 2020
\end{center}
% --- end date ---

\vspace{1cm}


% !split
\section{Neural networks}

Artificial neural networks are computational systems that can learn to
perform tasks by considering examples, generally without being
programmed with any task-specific rules. It is supposed to mimic a
biological system, wherein neurons interact by sending signals in the
form of mathematical functions between layers. All layers can contain
an arbitrary number of neurons, and each connection is represented by
a weight variable.

\subsection{Terminology}

Each time we describe a neural network algorithm we will typically specify three things. 

\begin{description}
\item[Architecture:] 
  The architecture specifies what variables are involved in the network and their topological relationships – for example, the variables involved in a neural net might be the weights of the connections between the neurons, along with the activities of the neurons.

\item[Activity rule:] 
  Most neural network models have short time-scale dynamics: local rules define how the activities of the neurons change in response to each other. Typically the activity rule depends on the weights (the parameters) in the network.

\item[Learning rule:] 
  The learning rule specifies the way in which the neural network’s weights change with time. This learning is usually viewed as taking place on a longer time scale than the time scale of the dynamics under the activity rule. Usually the learning rule will depend on the activities of the neurons. It may also depend on the values of target values supplied by a teacher and on the current value of the weights.
\end{description}

\noindent
% !split
\subsection{Artificial neurons}

The field of artificial neural networks has a long history of
development, and is closely connected with the advancement of computer
science and computers in general. A model of artificial neurons was
first developed by McCulloch and Pitts in 1943 to study signal
processing in the brain and has later been refined by others. The
general idea is to mimic neural networks in the human brain, which is
composed of billions of neurons that communicate with each other by
sending electrical signals.  Each neuron accumulates its incoming
signals, which must exceed an activation threshold to yield an
output. If the threshold is not overcome, the neuron remains inactive,
i.e.~has zero output.

This behaviour has inspired a simple mathematical model for an artificial neuron.
\begin{equation}
 y = f\left(\sum_{i=1}^n w_jx_j + b \right) = f(z),
 \label{artificialNeuron}
\end{equation}
where the bias $b$ is sometimes denoted $w_0$.
Here, the output $y$ of the neuron is the value of its activation function, which have as input
a weighted sum of signals $x_1, \dots ,x_n$ received by $n$ other neurons.

Conceptually, it is helpful to divide neural networks into four
categories:
\begin{enumerate}
\item general purpose neural networks, including deep neural networks (DNN) with several hidden layers, for supervised learning,

\item neural networks designed specifically for image processing, the most prominent example of this class being Convolutional Neural Networks (CNNs),

\item neural networks for sequential data such as Recurrent Neural Networks (RNNs), and

\item neural networks for unsupervised learning such as Deep Boltzmann Machines.
\end{enumerate}

\noindent
In natural science, DNNs and CNNs have already found numerous
applications. In statistical physics, they have been applied to detect
phase transitions in 2D Ising and Potts models, lattice gauge
theories, and different phases of polymers, or solving the
Navier-Stokes equation in weather forecasting.  Deep learning has also
found interesting applications in quantum physics. Various quantum
phase transitions can be detected and studied using DNNs and CNNs:
topological phases, and even non-equilibrium many-body
localization. 

In quantum information theory, it has been shown that one can perform
gate decompositions with the help of neural networks. 

The applications are not limited to the natural sciences. There is a
plethora of applications in essentially all disciplines, from the
humanities to life science and medicine.

% !split
\subsection{Neural network types}

An artificial neural network (ANN), is a computational model that
consists of layers of connected neurons, or nodes or units.  We will
refer to these interchangeably as units or nodes, and sometimes as
neurons.

It is supposed to mimic a biological nervous system by letting each
neuron interact with other neurons by sending signals in the form of
mathematical functions between layers.  A wide variety of different
ANNs have been developed, but most of them consist of an input layer,
an output layer and eventual layers in-between, called \emph{hidden
layers}. All layers can contain an arbitrary number of nodes, and each
connection between two nodes is associated with a weight variable.

Neural networks (also called neural nets) are neural-inspired
nonlinear models for supervised learning.  As we will see, neural nets
can be viewed as natural, more powerful extensions of supervised
learning methods such as linear and logistic regression and soft-max
methods we discussed earlier.


% !split
\paragraph{Feed-forward neural networks.}
The feed-forward neural network (FFNN) was the first and simplest type
of ANNs that were devised. In this network, the information moves in
only one direction: forward through the layers.

Nodes are represented by circles, while the arrows display the
connections between the nodes, including the direction of information
flow. Additionally, each arrow corresponds to a weight variable
(figure to come).  We observe that each node in a layer is connected
to \emph{all} nodes in the subsequent layer, making this a so-called
\emph{fully-connected} FFNN.



% !split
\paragraph{Convolutional Neural Network.}
A different variant of FFNNs are \emph{convolutional neural networks}
(CNNs), which have a connectivity pattern inspired by the animal
visual cortex. Individual neurons in the visual cortex only respond to
stimuli from small sub-regions of the visual field, called a receptive
field. This makes the neurons well-suited to exploit the strong
spatially local correlation present in natural images. The response of
each neuron can be approximated mathematically as a convolution
operation.  (figure to come)

Convolutional neural networks emulate the behaviour of neurons in the
visual cortex by enforcing a \emph{local} connectivity pattern between
nodes of adjacent layers: Each node in a convolutional layer is
connected only to a subset of the nodes in the previous layer, in
contrast to the fully-connected FFNN.  Often, CNNs consist of several
convolutional layers that learn local features of the input, with a
fully-connected layer at the end, which gathers all the local data and
produces the outputs. They have wide applications in image and video
recognition.

% !split
\paragraph{Recurrent neural networks.}
So far we have only mentioned ANNs where information flows in one
direction: forward. \emph{Recurrent neural networks} on the other hand,
have connections between nodes that form directed \emph{cycles}. This
creates a form of internal memory which are able to capture
information on what has been calculated before; the output is
dependent on the previous computations. Recurrent NNs make use of
sequential information by performing the same task for every element
in a sequence, where each element depends on previous elements. An
example of such information is sentences, making recurrent NNs
especially well-suited for handwriting and speech recognition.

% !split
\paragraph{Other types of networks.}
There are many other kinds of ANNs that have been developed. One type
that is specifically designed for interpolation in multidimensional
space is the radial basis function (RBF) network. RBFs are typically
made up of three layers: an input layer, a hidden layer with
non-linear radial symmetric activation functions and a linear output
layer (''linear'' here means that each node in the output layer has a
linear activation function). The layers are normally fully-connected
and there are no cycles, thus RBFs can be viewed as a type of
fully-connected FFNN. They are however usually treated as a separate
type of NN due the unusual activation functions.

% !split
\subsection{Multilayer perceptrons}

The \emph{multilayer perceptron} (MLP) is a very popular, and easy to implement approach, to deep learning. It consists of
\begin{enumerate}
\item a neural network with one or more layers of nodes between the input and the output nodes.

\item the multilayer network structure, or architecture, or topology, consists of an input layer, one or more hidden layers, and one output layer.

\item the input nodes pass values to the first hidden layer, its nodes pass the information on to the second and so on till we reach the output layer.
\end{enumerate}

\noindent
As a convention it is normal to call a network with one layer of input units, one layer of hidden units and one layer of output units as  a two-layer network. A network with two layers of hidden units is called a three-layer network etc.

The number of input nodes does not need to equal the number of output
nodes. This applies also to the hidden layers. Each layer may have its
own number of nodes and activation functions.

The hidden layers have their name from the fact that they are not
linked to observables and as we will see below when we define the
so-called activation $\boldsymbol{z}$, we can think of this as a basis
expansion of the original inputs $\boldsymbol{x}$. 

% !split
\paragraph{Why multilayer perceptrons?}
According to the \href{{http://citeseerx.ist.psu.edu/viewdoc/download?doi=10.1.1.441.7873&rep=rep1&type=pdf}}{universal approximation
theorem}, a feed-forward
neural network with just a single hidden layer containing a finite
number of neurons can approximate a continuous multidimensional
function to arbitrary accuracy, assuming the activation function for
the hidden layer is a \textbf{non-constant, bounded and
monotonically-increasing continuous function}. The theorem thus
states that simple neural networks can represent a wide variety of
interesting functions when given appropriate parameters. It is the
multilayer feedforward architecture itself which gives neural networks
the potential of being universal approximators.

Note that the requirements on the activation function only applies to
the hidden layer, the output nodes are always assumed to be linear, so
as to not restrict the range of output values.


% !split
\paragraph{Mathematical model.}
The output $y$ is produced via the activation function $f$
\[
 y = f\left(\sum_{i=1}^n w_ix_i + b \right) = f(z),
\]
This function receives $x_i$ as inputs.
Here the activation $z=(\sum_{i=1}^n w_ix_i+b)$. 
In an FFNN of such neurons, the \emph{inputs} $x_i$ are the \emph{outputs} of
the neurons in the preceding layer. Furthermore, an MLP is
fully-connected, which means that each neuron receives a weighted sum
of the outputs of \emph{all} neurons in the previous layer.

% !split
First, for each node $j$ in the first hidden layer, we calculate a weighted sum $z_j^1$ of the input coordinates $x_i$,

\begin{equation} z_j^1 = \sum_{i=1}^{n} w_{ji}^1 x_i + b_j^1
\end{equation}

Here $b_j^1$ is the so-called bias which is normally needed in
case of zero activation weights or inputs. How to fix the biases and
the weights will be discussed below.  The value of $z_j^1$ is the
argument to the activation function $f$ of each node $j$, The
variable $n$ stands for all possible inputs to a given node $j$ in the
first layer.  We define  the output $y_j^1$ of all neurons in layer 1 as

\begin{equation}
 y_j^1 = f(z_j^1) = f\left(\sum_{i=1}^n w_{ji}^1 x_i  + b_j^1\right),
 \label{outputLayer1}
\end{equation}
where we assume that all nodes in the same layer have identical
activation functions, hence the notation $f$. In general, we could assume in the more general case that different layers have different activation functions.
In this case we would identify these functions with a superscript $l$ for the $l$-th layer,

\begin{equation}
 y_i^l = f^l(z_i^l) = f^l\left(\sum_{j=1}^{N_{l-1}} w_{ij}^l y_j^{l-1} + b_i^l\right),
 \label{generalLayer}
\end{equation}
where $N_{l-1}$ is the number of nodes in layer $l-1$. When the output of
all the nodes in the first hidden layer are computed, the values of
the subsequent layer can be calculated and so forth until the output
is obtained.


% !split
The output of neuron $i$ in layer 2 is thus,
\begin{align}
 y_i^2 &= f^2\left(\sum_{j=1}^N w_{ij}^2 y_j^1 + b_i^2\right) \\
 &= f^2\left[\sum_{j=1}^N w_{ij}^2f^1\left(\sum_{k=1}^M w_{jk}^1 x_k + b_j^1\right) + b_i^2\right]
 \label{outputLayer2}
\end{align}
where we have substituted $y_k^1$ with the inputs $x_k$. Finally, the ANN output reads
\begin{align}
 y_i^3 &= f^3\left(\sum_{j=1}^N w_{ij}^3 y_j^2 + b_i^3\right) \\
 &= f^3\left[\sum_{j} w_{ij}^3 f^2\left(\sum_{k} w_{jk}^2 f^1\left(\sum_{m} w_{km}^1 x_m + b_k^1\right) + b_j^2\right)
  + b_1^3\right]
\end{align}

% !split
We can generalize this expression to an MLP with $L$ hidden
layers. The complete functional form is,

\begin{align}
&y^{L+1}_i = f^{L+1}\left[\!\sum_{j=1}^{N_L} w_{ij}^L f^L \left(\sum_{k=1}^{N_{L-1}}w_{jk}^{L-1}\left(\dots f^1\left(\sum_{n=1}^{N_0} w_{mn}^1 x_n+ b_m^1\right)\dots\right)+b_k^{L-1}\right)+b_1^L\right] &&
 \label{completeNN}
\end{align}
which illustrates a basic property of MLPs: The only independent
variables are the input values $x_n$.

% !split
This confirms that an MLP, despite its quite convoluted mathematical
form, is nothing more than an analytic function, specifically a
mapping of real-valued vectors $\boldsymbol{x} \in \mathbb{R}^n \rightarrow
\boldsymbol{y} \in \mathbb{R}^m$.

Furthermore, the flexibility and universality of an MLP can be
illustrated by realizing that the expression is essentially a nested
sum of scaled activation functions of the form

\begin{equation}
 f(x) = c_1 f(c_2 x + c_3) + c_4,
\end{equation}
where the parameters $c_i$ are weights and biases. By adjusting these
parameters, the activation functions can be shifted up and down or
left and right, change slope or be rescaled which is the key to the
flexibility of a neural network.

% !split
\paragraph{Matrix-vector notation.}
We can introduce a more convenient notation for the activations in an ANN. 

Additionally, we can represent the biases and activations
as layer-wise column vectors $\boldsymbol{b}_l$ and $\boldsymbol{y}_l$, so that the $i$-th element of each vector 
is the bias $b_i^l$ and activation $y_i^l$ of node $i$ in layer $l$ respectively. 

We have that $\boldsymbol{W}_l$ is an $N_{l-1} \times N_l$ matrix, while $\boldsymbol{b}_l$ and $\boldsymbol{y}_l$ are $N_l \times 1$ column vectors. 
With this notation, the sum becomes a matrix-vector multiplication, and we can write
the equation for the activations of hidden layer 2 (assuming three nodes for simplicity) as
\begin{equation}
 \boldsymbol{y}_2 = f_2(\boldsymbol{W}_2 \boldsymbol{y}_{1} + \boldsymbol{b}_{2}) = 
 f_2\left(\left[\begin{array}{ccc}
    w^2_{11} &w^2_{12} &w^2_{13} \\
    w^2_{21} &w^2_{22} &w^2_{23} \\
    w^2_{31} &w^2_{32} &w^2_{33} \\
    \end{array} \right] \cdot
    \left[\begin{array}{c}
           y^1_1 \\
           y^1_2 \\
           y^1_3 \\
          \end{array}\right] + 
    \left[\begin{array}{c}
           b^2_1 \\
           b^2_2 \\
           b^2_3 \\
          \end{array}\right]\right).
\end{equation}

% !split
\paragraph{Matrix-vector notation  and activation.}
The activation of node $i$ in layer 2 is

\begin{equation}
 y^2_i = f_2\Bigr(w^2_{i1}y^1_1 + w^2_{i2}y^1_2 + w^2_{i3}y^1_3 + b^2_i\Bigr) = 
 f_2\left(\sum_{j=1}^3 w^2_{ij} y_j^1 + b^2_i\right).
\end{equation}

This is not just a convenient and compact notation, but also a useful
and intuitive way to think about MLPs: The output is calculated by a
series of matrix-vector multiplications and vector additions that are
used as input to the activation functions. For each operation
$\mathrm{W}_l \boldsymbol{y}_{l-1}$ we move forward one layer.


% !split
\paragraph{Activation functions.}
A property that characterizes a neural network, other than its
connectivity, is the choice of activation function(s).  The following restrictions are imposed on an activation function for a FFNN to fulfill the universal approximation theorem

\begin{itemize}
  \item Non-constant

  \item Bounded

  \item Monotonically-increasing

  \item Continuous
\end{itemize}

\noindent
% !split
\paragraph{Logistic and Hyperbolic activation functions}
The second requirement excludes all linear functions. Furthermore, in
a MLP with only linear activation functions, each layer simply
performs a linear transformation of its inputs.

Regardless of the number of layers, the output of the NN will be
nothing but a linear function of the inputs. Thus we need to introduce
some kind of non-linearity to the NN to be able to fit non-linear
functions Typical examples are the logistic \emph{Sigmoid}

\[
 f_\mathrm{sigmoid}(z) = \frac{1}{1 + e^{-z}},
\]
and the \emph{hyperbolic tangent} function
\[
 f_\mathrm{tanh}(z) = \tanh(z)
\]

% !split
\paragraph{Rectifier activation functions}
The Rectifier Linear Unit (ReLU) uses the following activation function
\[
f_\mathrm{ReLU}(z) = \max(0,z).
\]

To solve a problem of dying ReLU neurons, practitioners often use a  variant of the ReLU
function, such as the leaky ReLU or the so-called
exponential linear unit (ELU) function
\[
f_\mathrm{ELU}(z) = \left\{\begin{array}{cc} \alpha\left( \exp{(z)}-1\right) & z < 0,\\  z & z \ge 0.\end{array}\right. 
\]



% !split
\paragraph{Relevance.}
The \emph{sigmoid} function are more biologically plausible because the
output of inactive neurons are zero. Such activation function are
called \emph{one-sided}. However, it has been shown that the hyperbolic
tangent performs better than the sigmoid for training MLPs. It has
become the most popular for \emph{deep neural networks}


% !split
\subsection{Deriving the back propagation code for a multilayer perceptron model}

Note: figures will be inserted later!

As we have seen the final output of a feed-forward network can be expressed in terms of basic matrix-vector multiplications.
The unknowwn quantities are our weights $w_{ij}$ and we need to find an algorithm for changing them so that our errors are as small as possible.
This leads us to the famous \href{{https://www.nature.com/articles/323533a0}}{back propagation algorithm}.

The questions we want to ask are how do changes in the biases and the
weights in our network change the cost function and how can we use the
final output to modify the weights?

To derive these equations let us start with a plain regression problem
and define our cost function as
\[
{\cal C}(\boldsymbol{W})  =  \frac{1}{2}\sum_{i=1}^n\left(y_i - t_i\right)^2, 
\]

where the $t_i$s are our $n$ targets (the values we want to
reproduce), while the outputs of the network after having propagated
all inputs $\boldsymbol{x}$ are given by $y_i$.  Other cost functions can also be considered.

% !split
\paragraph{Definitions.}
With our definition of the targets $\boldsymbol{t}$, the outputs of the
network $\boldsymbol{y}$ and the inputs $\boldsymbol{x}$ we
define now the activation $z_j^l$ of node/neuron/unit $j$ of the
$l$-th layer as a function of the bias, the weights which add up from
the previous layer $l-1$ and the forward passes/outputs
$\boldsymbol{a}^{l-1}$ from the previous layer as


\[
z_j^l = \sum_{i=1}^{M_{l-1}}w_{ij}^la_i^{l-1}+b_j^l,
\]

where $b_j^l$ are the biases from layer $l$.  Here $M_{l-1}$
represents the total number of nodes/neurons/units of layer $l-1$. The
figure here illustrates this equation.  We can rewrite this in a more
compact form as the matrix-vector products we discussed earlier,

\[
\boldsymbol{z}^l = \left(\boldsymbol{W}^l\right)^T\boldsymbol{a}^{l-1}+\boldsymbol{b}^l.
\]

With the activation values $\boldsymbol{z}^l$ we can in turn define the
output of layer $l$ as $\boldsymbol{a}^l = f(\boldsymbol{z}^l)$ where $f$ is our
activation function. In the examples here we will use the sigmoid
function discussed in the logistic regression lecture. We will also use the same activation function $f$ for all layers
and their nodes.  It means we have

\[
a_j^l = f(z_j^l) = \frac{1}{1+\exp{-(z_j^l)}}.
\]


% !split
\paragraph{Derivatives and the chain rule.}
From the definition of the activation $z_j^l$ we have
\[
\frac{\partial z_j^l}{\partial w_{ij}^l} = a_i^{l-1},
\]
and
\[
\frac{\partial z_j^l}{\partial a_i^{l-1}} = w_{ji}^l. 
\]

With our definition of the activation function we have (note that this function depends only on $z_j^l$)
\[
\frac{\partial a_j^l}{\partial z_j^{l}} = a_j^l(1-a_j^l)=f(z_j^l) \left[ 1-f(z_j^l) \right]. 
\]


% !split
\paragraph{Derivative of the cost function.}
With these definitions we can now compute the derivative of the cost function in terms of the weights.

Let us specialize to the output layer $l=L$. Our cost function is
\[
{\cal C}(\boldsymbol{W^L})  =  \frac{1}{2}\sum_{i=1}^n\left(y_i - t_i\right)^2=\frac{1}{2}\sum_{i=1}^n\left(a_i^L - t_i\right)^2, 
\]
The derivative of this function with respect to the weights is

\[
\frac{\partial{\cal C}(\boldsymbol{W^L})}{\partial w_{jk}^L}  =  \left(a_j^L - t_j\right)\frac{\partial a_j^L}{\partial w_{jk}^{L}}, 
\]
The last partial derivative can easily be computed and reads (by applying the chain rule)
\[
\frac{\partial a_j^L}{\partial w_{jk}^{L}} = \frac{\partial a_j^L}{\partial z_{j}^{L}}\frac{\partial z_j^L}{\partial w_{jk}^{L}}=a_j^L(1-a_j^L)a_k^{L-1},  
\]



% !split
\paragraph{Bringing it together, first back propagation equation.}
We have thus
\[
\frac{\partial{\cal C}(\boldsymbol{W^L})}{\partial w_{jk}^L}  =  \left(a_j^L - t_j\right)a_j^L(1-a_j^L)a_k^{L-1}, 
\]

Defining
\[
\delta_j^L = a_j^L(1-a_j^L)\left(a_j^L - t_j\right) = f'(z_j^L)\frac{\partial {\cal C}}{\partial (a_j^L)},
\]
and using the Hadamard product of two vectors we can write this as
\[
\boldsymbol{\delta}^L = f'(\boldsymbol{z}^L)\circ\frac{\partial {\cal C}}{\partial (\boldsymbol{a}^L)}.
\]

This is an important expression. The second term on the right handside
measures how fast the cost function is changing as a function of the $j$th
output activation.  If, for example, the cost function doesn't depend
much on a particular output node $j$, then $\delta_j^L$ will be small,
which is what we would expect. The first term on the right, measures
how fast the activation function $f$ is changing at a given activation
value $z_j^L$.

Notice that everything in the above equations is easily computed.  In
particular, we compute $z_j^L$ while computing the behaviour of the
network, and it is only a small additional overhead to compute
$f'(z^L_j)$.  The exact form of the derivative with respect to the
output depends on the form of the cost function.
However, provided the cost function is known there should be little
trouble in calculating

\[
\frac{\partial {\cal C}}{\partial (a_j^L)}
\]

With the definition of $\delta_j^L$ we have a more compact definition of the derivative of the cost function in terms of the weights, namely
\[
\frac{\partial{\cal C}(\boldsymbol{W^L})}{\partial w_{jk}^L}  =  \delta_j^La_k^{L-1}.
\]

% !split
\paragraph{Derivatives in terms of $z_j^L$.}
It is also easy to see that our previous equation can be written as

\[
\delta_j^L =\frac{\partial {\cal C}}{\partial z_j^L}= \frac{\partial {\cal C}}{\partial a_j^L}\frac{\partial a_j^L}{\partial z_j^L},
\]
which can also be interpreted as the partial derivative of the cost function with respect to the biases $b_j^L$, namely
\[
\delta_j^L = \frac{\partial {\cal C}}{\partial b_j^L}\frac{\partial b_j^L}{\partial z_j^L}=\frac{\partial {\cal C}}{\partial b_j^L},
\]
That is, the error $\delta_j^L$ is exactly equal to the rate of change of the cost function as a function of the bias. 

% !split

We have now three equations that are essential for the computations of the derivatives of the cost function at the output layer. These equations are needed to start the algorithm and they are


\begin{block_mdfboxadmon}[The starting equations]

\begin{equation}
\frac{\partial{\cal C}(\boldsymbol{W^L})}{\partial w_{jk}^L}  =  \delta_j^La_k^{L-1},
\end{equation}
and
\begin{equation}
\delta_j^L = f'(z_j^L)\frac{\partial {\cal C}}{\partial (a_j^L)},
\end{equation}
and

\begin{equation}
\delta_j^L = \frac{\partial {\cal C}}{\partial b_j^L},
\end{equation}
\end{block_mdfboxadmon} % title: The starting equations




An interesting consequence of the above equations is that when the
activation $a_k^{L-1}$ is small, the gradient term, that is the
derivative of the cost function with respect to the weights, will also
tend to be small. We say then that the weight learns slowly, meaning
that it changes slowly when we minimize the weights via say gradient
descent. In this case we say the system learns slowly.

Another interesting feature is that is when the activation function,
represented by the sigmoid function here, is rather flat when we move towards
its end values $0$ and $1$. In these
cases, the derivatives of the activation function will also be close
to zero, meaning again that the gradients will be small and the
network learns slowly again.



We need a fourth equation and we are set. We are going to propagate
backwards in order to determine the weights and biases. In order
to do so we need to represent the error in the layer before the final
one $L-1$ in terms of the errors in the final output layer.

% !split
\paragraph{Final back-propagating equation.}
We have that (replacing $L$ with a general layer $l$)
\[
\delta_j^l =\frac{\partial {\cal C}}{\partial z_j^l}.
\]
We want to express this in terms of the equations for layer $l+1$. Using the chain rule and summing over all $k$ entries we have

\[
\delta_j^l =\sum_k \frac{\partial {\cal C}}{\partial z_k^{l+1}}\frac{\partial z_k^{l+1}}{\partial z_j^{l}}=\sum_k \delta_k^{l+1}\frac{\partial z_k^{l+1}}{\partial z_j^{l}},
\]
and recalling that
\[
z_j^{l+1} = \sum_{i=1}^{M_{l}}w_{ij}^{l+1}a_i^{l}+b_j^{l+1},
\]
with $M_l$ being the number of nodes in layer $l$, we obtain
\[
\delta_j^l =\sum_k \delta_k^{l+1}w_{kj}^{l+1}f'(z_j^l),
\]
This is our final equation.

We are now ready to set up the algorithm for back propagation and learning the weights and biases.

% !split
\subsection{Setting up the back-propagation algorithm}



The four equations  provide us with a way of computing the gradient of the cost function. Let us write this out in the form of an algorithm.


\begin{summary_mdfboxadmon}[Summary]
\begin{itemize}
\item First, we set up the input data $\boldsymbol{x}$ and the activations $\boldsymbol{z}_1$ of the input layer and compute the activation function and the outputs $\boldsymbol{a}^1$.

\item Secondly, perform the feed-forward until we reach the output layer. I.e., compute all activation functions and the pertinent outputs $\boldsymbol{a}^l$ for $l=2,3,\dots,L$.

\item Compute the ouput error $\boldsymbol{\delta}^L$ by
\end{itemize}

\noindent
\[
\delta_j^L = f'(z_j^L)\frac{\partial {\cal C}}{\partial (a_j^L)}.
\]

\begin{itemize}
\item Back-propagate the error for each $l=L-1,L-2,\dots,2$ as
\end{itemize}

\noindent
\[
\delta_j^l = \sum_k \delta_k^{l+1}w_{kj}^{l+1}f'(z_j^l).
\]

\begin{itemize}
\item Finally, update the weights and the biases using gradient descent for each $l=L-1,L-2,\dots,2$ and update the weights and biases according to the rules
\end{itemize}

\noindent
\[
w_{jk}^l\leftarrow  w_{jk}^l- \eta \delta_j^la_k^{l-1},
\]

\[
b_j^l \leftarrow b_j^l-\eta \frac{\partial {\cal C}}{\partial b_j^l}=b_j^l-\eta \delta_j^l,
\]
\end{summary_mdfboxadmon} % title: Summary



The parameter $\eta$ is the learning rate.
Here it is convenient to use stochastic gradient descent with mini-batches and an outer loop that steps through multiple epochs of training.


% !split 
\subsection{Learning challenges}

The back-propagation algorithm works by going from
the output layer to the input layer, propagating the error gradient. The learning algorithm uses these
gradients to update each parameter with a Gradient Descent (GD) step.

Unfortunately, the gradients often get smaller and smaller as the
algorithm progresses down to the first hidden layers. As a result, the
GD update step leaves the lower layer connection weights
virtually unchanged, and training never converges to a good
solution. This is known in the literature as 
\textbf{the vanishing gradients problem}. 

In other cases, the opposite can happen, namely that the gradients grow bigger and
bigger. The result is that many of the layers get large updates of the 
weights and the learning algorithm diverges. This is the \textbf{exploding gradients problem}, which is mostly encountered in recurrent neural networks. More generally, deep neural networks suffer from unstable gradients, different layers may learn at widely different speeds

% !split 
\paragraph{Is the Logistic activation function (Sigmoid) our choice?}
Although this unfortunate behavior has been empirically observed for
quite a while (it was one of the reasons why deep neural networks were
mostly abandoned for a long time), it is only around 2010 that
significant progress was made in understanding it.

A paper titled \href{{http://proceedings.mlr.press/v9/glorot10a.html}}{Understanding the Difficulty of Training Deep
Feedforward Neural Networks by Xavier Glorot and Yoshua Bengio} identified problems with the popular logistic
sigmoid activation function and the weight initialization technique
that was most popular at the time (namely random initialization using
a normal distribution with a mean of 0 and a standard deviation of
1). 

They showed that with this activation function and this
initialization scheme, the variance of the outputs of each layer is
much greater than the variance of its inputs. Going forward in the
network, the variance keeps increasing after each layer until the
activation function saturates at the top layers. This is actually made
worse by the fact that the logistic function has a mean of 0.5, not 0
(the hyperbolic tangent function has a mean of 0 and behaves slightly
better than the logistic function in deep networks).


% !split
\paragraph{The derivative of the Logistic funtion.}
Looking at the logistic activation function, when inputs become large
(negative or positive), the function saturates at 0 or 1, with a
derivative extremely close to 0. Thus when backpropagation kicks in,
it has virtually no gradient to propagate back through the network,
and what little gradient exists keeps getting diluted as
backpropagation progresses down through the top layers, so there is
really nothing left for the lower layers.

In their paper, Glorot and Bengio proposed a way to significantly
alleviate this problem. The signal must flow properly in both
directions: in the forward direction when making predictions, and in
the reverse direction when backpropagating gradients. We don’t want
the signal to die out, nor do we want it to explode and saturate. For
the signal to flow properly, the authors argue that we need the
variance of the outputs of each layer to be equal to the variance of
its inputs, and we also need the gradients to have equal variance
before and after flowing through a layer in the reverse direction.



One of the insights in the 2010 paper by Glorot and Bengio was that
the vanishing/exploding gradients problems were in part due to a poor
choice of activation function. Until then most people had assumed that
if Nature had chosen to use roughly sigmoid activation functions in
biological neurons, they must be an excellent choice. But it turns out
that other activation functions behave much better in deep neural
networks, in particular the ReLU activation function, mostly because
it does not saturate for positive values (and also because it is quite
fast to compute).


% !split
\paragraph{The RELU function family.}
The Rectifier Linear Unit (ReLU) uses the following activation function
\[
f(z) = \max(0,z).
\]

The ReLU activation function suffers from a problem known as the dying
ReLUs: during training, some neurons effectively die, meaning they
stop outputting anything other than 0.

In some cases, you may find that half of your network’s neurons are
dead, especially if you used a large learning rate. During training,
if a neuron’s weights get updated such that the weighted sum of the
neuron’s inputs is negative, it will start outputting 0. When this
happen, the neuron is unlikely to come back to life since the gradient
of the ReLU function is 0 when its input is negative.

To solve this problem, nowadays practitioners use a  variant of the ReLU
function, such as the leaky ReLU discussed above or the so-called
exponential linear unit (ELU) function

\[
ELU(z) = \left\{\begin{array}{cc} \alpha\left( \exp{(z)}-1\right) & z < 0,\\  z & z \ge 0.\end{array}\right. 
\]

% !split
\paragraph{Which activation function should we use?}
In general it seems that the ELU activation function is better than
the leaky ReLU function (and its variants), which is better than
ReLU. ReLU performs better than $\tanh$ which in turn performs better
than the logistic function. 

If runtime
performance is an issue, then you may opt for the  leaky ReLU function  over the 
ELU function If you don’t
want to tweak yet another hyperparameter, you may just use the default
$\alpha$ of $0.01$ for the leaky ReLU, and $1$ for ELU. If you have
spare time and computing power, you can use cross-validation or
bootstrap to evaluate other activation functions.


% !split 
\subsection{A top-down perspective on Neural networks}

The first thing we would like to do is divide the data into two or three
parts. A training set, a validation or dev (development) set, and a
test set. 
\begin{itemize}
\item The training set is used for learning and adjusting the weights.

\item The dev/validation set is a subset of the training data. It is used to
\end{itemize}

\noindent
check how well we are doing out-of-sample, after training the model on
the training dataset. We use the validation error as a proxy for the
test error in order to make tweaks to our model, e.g.~changing hyperparameters such as the learning rate.
\begin{itemize}
\item The test set will be used to test the performance of or predictions with the final neural net. 
\end{itemize}

\noindent
It is crucial that we do not use any of the test data to train the algorithm. This is a cardinal sin in ML. T If the validation and test sets are drawn from the same distributions, then a good performance on the validation set should lead to similarly good performance on the test set. 

However, sometimes
the training data and test data differ in subtle ways because, for
example, they are collected using slightly different methods, or
because it is cheaper to collect data in one way versus another. In
this case, there can be a mismatch between the training and test
data. This can lead to the neural network overfitting these small
differences between the test and training sets, and a poor performance
on the test set despite having a good performance on the validation
set. To rectify this, Andrew Ng suggests making two validation or dev
sets, one constructed from the training data and one constructed from
the test data. The difference between the performance of the algorithm
on these two validation sets quantifies the train-test mismatch. This
can serve as another important diagnostic when using DNNs for
supervised learning.

% !split
\subsection{Limitations of supervised learning with deep networks}

Like all statistical methods, supervised learning using neural
networks has important limitations. This is especially important when
one seeks to apply these methods, especially to physics problems. Like
all tools, DNNs are not a universal solution. Often, the same or
better performance on a task can be achieved by using a few
hand-engineered features (or even a collection of random
features). 

Here we list some of the important limitations of supervised neural network based models. 



\begin{itemize}
\item \textbf{Need labeled data}. All supervised learning methods, DNNs for supervised learning require labeled data. Often, labeled data is harder to acquire than unlabeled data (e.g.~one must pay for human experts to label images).

\item \textbf{Supervised neural networks are extremely data intensive.} DNNs are data hungry. They perform best when data is plentiful. This is doubly so for supervised methods where the data must also be labeled. The utility of DNNs is extremely limited if data is hard to acquire or the datasets are small (hundreds to a few thousand samples). In this case, the performance of other methods that utilize hand-engineered features can exceed that of DNNs.

\item \textbf{Homogeneous data.} Almost all DNNs deal with homogeneous data of one type. It is very hard to design architectures that mix and match data types (i.e.~some continuous variables, some discrete variables, some time series). In applications beyond images, video, and language, this is often what is required. In contrast, ensemble models like random forests or gradient-boosted trees have no difficulty handling mixed data types.

\item \textbf{Many problems are not about prediction.} In natural science we are often interested in learning something about the underlying distribution that generates the data. In this case, it is often difficult to cast these ideas in a supervised learning setting. While the problems are related, it is possible to make good predictions with a \emph{wrong} model. The model might or might not be useful for understanding the underlying science.
\end{itemize}

\noindent
Some of these remarks are particular to DNNs, others are shared by all supervised learning methods. This motivates the use of unsupervised methods which in part circumvent these problems.



% ------------------- end of main content ---------------

% #ifdef PREAMBLE
\end{document}
% #endif

